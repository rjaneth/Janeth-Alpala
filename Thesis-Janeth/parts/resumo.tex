\begin{otherlanguage}{portuguese} 


\begin{resumo}
\phantomsection
\addcontentsline{toc}{chapter}{Resumo}
%\addchaptertocentry{\resumenname} % Add the abstract to the table of contents
A detecção de bordas tem um papel essencial no pós-processamento das imagens PolSAR. A extração de todas as características das bordas e a supressão de ruídos speckle, especialmente quando bordas fracas/fortes aparecem simultaneamente dentro e fora de áreas heterogêneas, ainda é um grande desafio.

As imagens PolSAR podem fornecer mais informações do que as imagens de radar de abertura sintética (SAR) de polarimetria única. Como a etapa de pré-requisito do processamento de imagem, a detecção de bordas PolSAR é muito importante, o que pode fornecer informações estruturais importantes para reconhecimento adicional de objetos e interpretação de imagens PolSAR. No entanto, uma cena PolSAR complexa geralmente inclui tipos de terreno heterogêneos e homogêneos, como áreas urbanas, florestas, terras agrícolas, águas e assim por diante.

Nesta tese, obtemos as propriedades estatísticas (bias, variância) dos estimadores de pontos de borda em imagens SAR/PolSAR. Desta forma, propomos e avaliamos novas técnicas de fusão e seleção de evidências que levem em consideração essas propriedades.

\end{resumo}
\end{otherlanguage}