\chapter{Introduction}\label{chp:int}
%------------------------------------
%	INTRO INTRO
%------------------------------------

The use of Statistical Information Theory (SIT) and Statistical Information Geometry (SIG) in image processing and analysis problems has the potential to provide several solutions to the same problem. Considering that these solutions are evidences, or estimates of the solution, this project aims to propose techniques for fusing them into a single solution potentially better than the individual evidences.

The starting point is the paper by de Borba et al. \cite{de2020fusion}, in which the authors obtain several evidences of the location of edge points in images with low signal/noise relation. In this work, the estimations are treated deterministically, that is, without taking into account the variability of the estimator that produces them. These estimations go through fusion processes to obtain an edge point that summarizes the evidence.

The project proposes to create fusion techniques that take into account the quality of each evidence. In this way, evidence that is subject to high variability, or that is inconsistent with most other evidence, will have less influence on the final result.

On the one hand, the project is challenging because there are no theoretical results to assess the quality (bias, variance, etc.) of the evidence. On the other hand, when they come from an analysis based on the SIT/GIS approach, there is the possibility of advancing the frontier of knowledge since there is a powerful statistical tool (although little explored) for this purpose.

An explainable fusion of this type of evidence will be able to provide semantically valuable information on the contribution of each component to the composition of the final result. With this, end-users will be able to extract relevant information about the information content (and reliability) of each source of evidence. This knowledge will allow, for example, discarding unreliable sources, or those that, being redundant, contribute little to the quality of the product.

The work will focus on SAR/PolSAR (Synthetic Aperture Radar/Polarimetric SAR) images. These images are of great relevance in Remote Sensing \cite{cloude2009polarisation}, and present high levels of non-Gaussian and non-additive noise. This last feature makes them attractive and challenging for the development of new techniques \cite{lee2017polarimetric}.

The statistical approach to edge detection in SAR/PolSAR images has provided comparable or better results than techniques previously considered state-of-the-art. Among the edge detection results to be highlighted for SAR images (in which each pixel has a non-negative value), we mention the works of \cite{gambini2008accuracy, giron2012nonparametric}.


\section{Literature Review}
A. Fink \cite{fink2019conducting} provides the following definition: ``A literature review is a systematic, explicit, and reproducible project to identify, evaluate, and interpret the existing body of recorded documents.''  Literature reviews generally aim at two goals: first, they summarize existing research by identifying patterns, themes, and issues. Second, this helps identify the conceptual content of the field and can contribute to theory development.

\subsection{Search Strategy}

In order to select the most relevant articles for this study, we started by searching for papers on edge detection in SAR/PolSAR images in the most common scientific databases: Scopus, Web of Science (WoS), Science Direct (SD), Google Scholar (GS), Emerald insight (Emerald), Wiley Online Library (Wiley), Taylor \& Francis Online (T\&F), Springer Link (Springer), Inderscience (IS), and Informs PubsOnline (IPO). \\ %The Search Terms employed in the field of Title are the following:
The employed keywords were the following focused on the fields Title-Abstract-Keywords:
\begin{itemize}
	\item ("bias correction" OR "improve estimators" OR "new estimator" OR "MLE" OR "efficient estimators" OR "maximum-likelihood estimation" OR "statistical information theory")  
\item	AND ("boundary detection" OR "edge detection" OR "edge detection method" OR "edge detector") 
\item ("PolSAR images" OR "Polarimetric Synthetic Aperture Radar" OR "single-polarimetric synthetic aperture radar images" OR "SAR images")
\end{itemize}
Some works related to this topic can be seen in Table \ref{tab:1}.
\begin{longtable}{P{\dimexpr0.26\textwidth-2\tabcolsep-2\arrayrulewidth\relax}
                        P{\dimexpr0.39\textwidth-2\tabcolsep-\arrayrulewidth\relax}
                        P{\dimexpr0.30\textwidth-2\tabcolsep-\arrayrulewidth\relax}
                        %P{\dimexpr0.20\textwidth-2\tabcolsep-\arrayrulewidth\relax}
                        }
\toprule
Author & Title & Source \\\midrule
\rowcolor{Gray}\multicolumn{3}{c}{Edge detection}\\
Nascimento, A. D., Horta, M. M., Frery, A. C., and Cintra, R. J. (2013)\cite{nascimento2013comparing}. & Comparing edge detection methods based on stochastic entropies and distances for PolSAR imagery & IEEE journal of selected topics in applied earth observations and remote sensing\\ \midrule
%\rowcolor{gray!20}\multicolumn{3}{c}{Music}\\
Xiang, Y., Wang, F., Wan, L., and You, H. (2017) \cite{xiang2017sar} & SAR-PC: Edge detection in SAR images via an advanced phase congruency model. & Remote Sensing\\ \midrule

Qin, X., Hu, T., Yu, W., Wang, P., Li, J.,  Zou, H. (2018) \cite{qin2018edge}& Edge detection of PolSAR images using statistical distance between automatically refined samples.& IEEE International Geoscience and Remote Sensing Symposium\\
\bottomrule
Nascimento, A. D., Frery, A. C.,  Cintra, R. J. (2019) \cite{nascimento2018detecting}& Detecting changes in fully polarimetric SAR imagery with statistical information theory.& IEEE Transactions on Geoscience and Remote Sensing\\
\bottomrule
Quan, S., Xiang, D., Xiong, B.,  Kuang, G. (2020) \cite{quan2019edge}& Edge detection for PolSAR images integrating scattering characteristics and optimal contrast.& IEEE Geoscience and Remote Sensing Letters\\
\bottomrule
\caption[Papers related to edge detection]{Papers related to edge detection}\label{tab:1}
%\caption{\label{Pesquisa de documentos.} works that use Machine Learning}
\end{longtable}




%("REVIEW" OR "STATE OF THE ART" OR "SURVEY" OR "CONCEPTUAL FRAMEWORK" ) AND ("LAYOUT DESIGN" OR "DESIGN OF LAYOUT" OR "MANUFACTURING LAYOUT"
%OR "FACILITY LAYOUT" OR "PLANT LAYOUT" OR "FACILITY LAYOUT PLANNING" ). 

%---------------------------------------------------------------------------------------------------------------------
\section{Project objectives}
The objective of this thesis is to contribute to the state of the art in the fusion of evidence using a statistical approach to achieve explainable and relevant results. \\

The main specific objectives are:

\begin{itemize}
	\item Obtain the statistical properties (bias, variance) of edge point estimators in SAR/PolSAR images.
	\item Propose and evaluate new fusion and evidence selection techniques that take these properties into account.
	\item Develop new features for production software, e.g. PolSARpro and SNAP.
\end{itemize}

\section{Thesis organization}\label{sec:research_questions}

% colocar para citar los capitulos
\hypersetup{linkcolor=blue}

This thesis is structured in 5 chapters. In the upcoming sections of the current chapter a general overview on ... \\
 Chapter~\ref{chp:obs} details ...\\
Chapter ~\ref{chp:models} explains ...
%------------------------------------
%	HAZARD: OVERVIEW
%------------------------------------


\section{Computing platforms}
This  project required a vast amount of computing, preprocessing, analysis and plotting.

%------------------------------------
%	TODO STUFF
%------------------------------------
% \section{Additional sources to integrate in this chapter}
% \begin{itemize}
% \item check all TODOs
% \item read again \url{http://ec.europa.eu/environment/water/flood_risk/flood_atlas/pdf/handbook_goodpractice.pdf}
% \item include citation EU Floods Directive (2007/60/EC) , \cite{Mysiak2013, EUFD2007}
% \item section to talk about uncertainties? \citet{alfieri2014} has a good section about it
% \item Alps are the water tower of the Po plain, and for this reason I might want to give them a closer look. I could cite one of the many kotlarski works, or e.g. \citet[][]{Gobiet2014}. View the 'Alps' category in my Mendeley, it contains about 30 works on the subject
% \end{itemize}