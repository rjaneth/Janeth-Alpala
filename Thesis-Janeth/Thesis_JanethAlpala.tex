%% Template para dissertação/tese na classe UFPEthesis
%% versão 0.9.2
%% (c) 2005 Paulo G. S. Fonseca
%% www.cin.ufpe.br/~paguso/ufpethesis

%% Carrega a classe ufpethesis
%% Opções: * Idiomas
%%           pt   - português (padrão)
%%           en   - inglês
%%         * Tipo do Texto
%%           bsc  - para monografias de graduação
%%           msc  - para dissertações de mestrado (padrão)
%%           qual - exame de qualificação doutorado
%%           prop - proposta de tese doutorado
%%           phd  - para teses de doutorado
%%         * Mídia
%%           scr  - para versão eletrônica (PDF) / consulte o guia do usuario
%%         * Estilo
%%           classic - estilo original à la TAOCP
%%           modern  - estilo à la CUP (padrão)
%%           ugly    - formato da UFPE com parte pré-textual no formato ABNT
%%         * Paginação
%%           oneside - para impressão em face única
%%           twoside - para impressão em frente e verso (padrão)
\documentclass{ufpethesis}
% Paquetes LaTeX y estilos globales
%\usepackage[utf8]{inputenc}
%\usepackage[portugues,english]{babel}
%\usepackage[portuguese,english]{babel}
\usepackage{multicol}
%\usepackage{xcolor}
%\usepackage{subfigure}
%\usepackage[spanish]{babel}

\usepackage{graphicx}
%\usepackage{color}% para colocar la tabla
% color package for formatting URLs 
\usepackage[dvipsnames]{xcolor}
%\usepackage{float}
\usepackage{titlesec}
\usepackage[bookmarks,breaklinks,colorlinks=true,  allcolors=black, citecolor=BlueViolet]{hyperref}
%\hypersetup{colorlinks=true, linkcolor=BrickRed, urlcolor=RoyalBlue, citecolor=BlueViolet}
%\urlstyle{same}


%\usepackage[hidelinks,colorlinks=true,linkcolor=blue,citecolor=blue]{hyperref}
\usepackage{listings}
\usepackage{inconsolata}
\usepackage{float}

%\usepackage[square,numbers]{natbib}
%\AtBeginDocument{
%  \renewcommand{\bibsection}{\chapter{\bibname}}
%} % Bibliografia en capitulo numerado

%\usepackage{geometry}
\usepackage{amsmath,amsthm}% cambia la fuente de la letra y ecuaciones con amsfonts,
%\usepackage{amsmath}  % For math
\usepackage{amssymb}  % For more math

\usepackage{parskip}
%\usepackage[official]{eurosym}
\usepackage{todonotes}
\usepackage{csquotes}

% new fuente
\usepackage{tgtermes}
%\usepackage{float}
%\usepackage{bm}
\usepackage{graphicx} % Required for including pictures
\usepackage{mathtools}
%\usepackage{multicol}
%\usepackage{caption} % For caption spacing
%\usepackage{subcaption} % For sub-figures

%\usepackage{booktabs} % better tables
\usepackage{rotating} % landscape stuff, such as tables
\usepackage{array} % for m columns
\usepackage{multirow} % multi row tables
\usepackage{booktabs}
\usepackage{dcolumn}% for align columns
%\usepackage{array} % space rows
\usepackage{tabu}
%\usepackage{bigdelim} % for big braces in tables (https://tex.stackexchange.com/a/129797/66561)
%\usepackage{subcaption}
% Subfigures
\usepackage{mwe}
% The old way:
%\usepackage[labelformat=simple]{subcaption}
%\makeatletter
%\renewcommand*{\thesubfigure}{(\alph{subfigure})}
%\renewcommand*{\p@subfigure}{}
%\makeatother
% The new way:
\usepackage{subcaption}
\labelformat{subfigure}{(#1)}% see ltnews30.pdf

%\usepackage{afterpage} % for using \afterpage{\clearpage} before sidewaysfigures, to prevent them from going to the end:
%https://latex.org/forum/viewtopic.php?t=5903

%\usepackage[skip=-2pt, position=top, labelfont=normalfont]{subcaption} % multiple figures in one
%skip=-2pt reduces space between caption and subfigure
%singlelinecheck=false, justification=raggedright move the caption to the left

%\usepackage{microtype}%micro-typographic extension for better looks (subliminal refinements towards typographical perfection)
%%\usepackage{textcomp}% Solves some warnings....
%\usepackage[titletoc, title, header]{appendix}% For nicer appendices

%\usepackage{tabularx} % better tables

%\usepackage{multicol} % itemizations on multiple columns

%\newcommand{\HRule}{\rule{.9\linewidth}{.6pt}} % New command to make the lines in the title page
\newcommand{\decoRule}{\rule{.8\textwidth}{.4pt}} % New command for a rule to be used under figures
%\newcommand{\halfDecoRule}{\rule{.4\textwidth}{.4pt}} % New command for a rule to be used under figures

% con el comando \par empieza un nuevo parrafo con sangria, y con \\ comienza una nueva línea pero no un nuevo párrafo.
% las subsecciones empiezan sin sangria \parindent \indent
%\noindent to remove
% se establecerá en cero cuando se inserte un encabezado de sección
%\usepackage{indentfirst}
%set title spacing
%\usepackage{indentfirst}

\setlength\parindent{24pt}
\setlength{\parskip}{3.0pt}
\titlespacing*{\section}{0pt}{3.5ex plus 1ex minus .2ex}{2.3ex plus .2ex}
\titlespacing*{\subsection}{0pt}{3.5ex plus 1ex minus .2ex}{2.3ex plus .2ex}
%\usepackage{parskip}
%\setstretch{1.20}

% fancy chapter
\usepackage[Lenny]{fncychap}

 \usepackage{bm,bbm}
    \usepackage{mathrsfs}
    \usepackage{siunitx}
    %\usepackage{graphicx}
   \usepackage{url}
    %\usepackage[T1]{fontenc}
    \usepackage{polski}
   % \usepackage{booktabs}
    \usepackage{color}
   \usepackage{xcolor}
	\RequirePackage{microtype} % For command \textls[]{}
   % \usepackage{amsmath}
   %\usepackage{multirow}
	
	% operator names
	
\newcommand{\bias}{\operatorname{Bias}}
\newcommand{\widebar}[1]{\overline{#1}}
%\newcommand{\widebar}[1]{\overline{#1}}
\DeclareFontFamily{U}{mathx}{}
\DeclareFontShape{U}{mathx}{m}{n}{<-> mathx10}{}
\DeclareSymbolFont{mathx}{U}{mathx}{m}{n}
%\DeclareMathAccent{\widehat}{0}{mathx}{"70}
\DeclareMathAccent{\widecheck}{0}{mathx}{"71}

%\usepackage{enumitem}%abreviaciones
%	HEADERS AND FOOTERS
%----------------------------------------------------------------------------------------
% abbreviations
%	ABBREVIATIONS PAGE DESIGN
%----------------------------------------------------------------------------------------
%--------------------------References
 %\usepackage[backend=biber, style=numeric, citestyle=nature]{biblatex}
 
% \usepackage{csquotes}
 % \usepackage[babel]{csquotes}
%\usepackage[style=numeric, maxbibnames=99, giveninits]{biblatex}

 % giveninits : es para colocar solo la inicial del nombre
 % giveninits : es para colocar solo la inicial del nombre

%\usepackage[style=numeric-comp, sorting=none, backend=bibtex, doi=false, isbn=false, natbib=true, giveninits]{biblatex} % sorting=none ordena por orden de citas
%%\bibliography{../Common/references}
%\addbibresource{../Common/references.bib}
%%\bibliography{/Volumes/cld/Dropbox/Articles/references}

\usepackage[style=authoryear, sorting=nyt, backend=bibtex, natbib=true, backref=true, doi=false, isbn=false, maxcitenames=1, maxbibnames=99, giveninits]{biblatex}
\addbibresource{../Common/references.bib}
\DeclareNameAlias{sortname}{family-given}
\setlength{\bibitemsep}{\parskip}


%% Redefinir el formato de las citas
%\DeclareCiteCommand{\cite}
  %{\usebibmacro{prenote}}
  %{\usebibmacro{citeindex}%
   %\printnames{labelname}%
   %\setunit{\addspace}%
   %\mkbibparens{\printfield{year}}}
  %{\multicitedelim}
  %{\usebibmacro{postnote}}
%
%% Configurar hipervínculos para nombres de autor y año
%\DeclareFieldFormat{citehyperref}{%
  %\DeclareFieldAlias{bibhyperref}{noformat}% Evitar que los nombres de autor estén vinculados
  %\bibhyperref{#1}}
%\setlength\bibitemsep{\itemsep}
%"\setlength{\bibitemsep}{\parskip}% para colocar espacio entre los items de referencias
% Redefinir el formato de las citas
%\DeclareCiteCommand{\cite}
  %{\usebibmacro{prenote}}
  %{\usebibmacro{citeindex}%
   %\printtext[bibhyperref]{(\printfield{year})}}
  %{\multicitedelim}
  %{\usebibmacro{postnote}}
% para colocar la tabla de revision bibliografica
%\usepackage{osameet3}
\usepackage{booktabs, longtable,array}
\usepackage{color, colortbl}
%\definecolor{name}{system}{definition}
\definecolor{Gray}{gray}{0.9}
%\usepackage[table]{xcolor}
\newcolumntype{P}[1]{>{\raggedright\arraybackslash}p{#1}}
%\input{etc/style}

%% Preâmbulo:
%% coloque aqui o seu preâmbulo LaTeX, i.e., declaração de pacotes,
%% (re)definições de macros, medidas, etc.

%% Identificação:

% Universidade
% e.g. \university{Universidade de Campinas}
% Na UFPE, comente a linha a seguir
\university{Universidade Federal de Pernambuco}

% Modifique o comando \universitylogo para alterar o logo da universidade
% e.g.
% \renewcommand{\universitylogo}{\includegraphics{newlogo.pdf}}

% Endereço (cidade)
% e.g. \address{Campinas}
% Na UFPE, comente a linha a seguir
\address{Recife}

% Instituto ou Centro Acadêmico
% e.g. \institute{Centro de Ciências Exatas e da Natureza}
% Comente se não se aplicar
\institute{Centro de Ciências Exatas e da Natureza}

% Departamento acadêmico
% e.g. \department{Departamento de Informática}
% Comente se não se aplicar
%\department{Programa de
%Pós-graduação em Estatística}

 

% e.g. \program{Pós-graduação em Ciência da Computação}
\program{Pós-graduação em Estatística}

% Área de titulação
% e.g. \majorfield{Ciência da Computação}
%\majorfield{Programa de
%Pós-graduação em Estatística}

% Título da dissertação/tese
% e.g. \title{Sobre a conjectura $P=NP$}
\title{Fusão Explicável de Evidências Estatísticas de Bordas em Imagens}

% Data da defesa
% e.g. \date{19 de fevereiro de 2003}
\date{2022}

% Autor
% e.g. \author{José da Silva}
\author{Rosa Janeth Alpala}

% Orientador(a)
% Opção: [f] - para orientador do sexo feminino
% e.g. \adviser[f]{Profa. Dra. Maria Santos}
\adviser{Dr. Alejandro C. Frery}

% Orientador(a)
% Opção: [f] - para orientador do sexo feminino
% e.g. \coadviser{Prof. Dr. Pedro Pedreira}
% Comente se não se aplicar
%\coadviser{NOME DO(DA) CO-ORIENTADOR(A)}

%% Inicio do documento

%\setlength\headheight{15pt}

%colocar linea en la cabecera
%\usepackage{fancyhdr}
%\fancyhf{}
%\fancyhead[CE]{\small\slshape\leftmark}
%\fancyhead[CO]{\small\slshape\rightmark}
%\fancyfoot[CE,CO]{\small\thepage}
%\renewcommand{\headrulewidth}{0.4pt}
%\renewcommand{\footrulewidth}{0.4pt}


%\pagestyle{fancy} %coloca la linea en la cabecera


\begin{document}

%%
%% Parte pré-textual
%%

\frontmatter
% Folha de rosto
% Comente para ocultar
\frontpage

% Portada (apresentação)
% Comente para ocultar
%\pagestyle{empty}
%\pagenumbering{gobble}
%\pagenumbering{alph}
%\thispagestyle{empty}
\cleardoublepage
    %Abstract - begin
        \begingroup
        \let\clearpage\relax
        \let\cleardoublepage\relax
        \let\cleardoublepage\relax

\presentationpage %<======= subportada
\thispagestyle{empty}%<=======
        \endgroup           
        \vfill 
\newpage
%\thispagestyle{empty}
%\frontmatter
%\cleardoublepage% \clearpage
%\pagenumbering{roman}

% Dedicatória
% Comente para ocultar
%\begin{dedicatory}
%DIGITE A DEDICATÓRIA AQUI
%\end{dedicatory}

% Agradecimentos
% Se preferir, crie um arquivo à parte e o inclua via \include{}
%\acknowledgements
%DIGITE OS AGRADECIMENTOS AQUI

% Epígrafe
% Comente para ocultar
% e.g.
%  \begin{epigraph}[Tarde, 1919]{Olavo Bilac}
%  Última flor do Lácio, inculta e bela,\\
%  És, a um tempo, esplendor e sepultura;\\
%  Ouro nativo, que, na ganga impura,\\
%  A bruta mina entre os cascalhos vela.
%  \end{epigraph}
%\begin{epigraph}[NOTA]{AUTOR}
%DIGITE AQUI A CITAÇÃO
%\end{epigraph}

% Resumo em Português
% Se preferir, crie um arquivo à parte e o inclua via \include{}
%\resumo
%DIGITE O RESUMO AQUI
% Palavras-chave do resumo em Português
%\begin{keywords}
%DIGITE AS PALAVRAS-CHAVE AQUI
%\end{keywords}

% Resumo em Inglês
% Se preferir, crie um arquivo à parte e o inclua via \include{}

%\newpage
%\pagenumbering{arabic}

\setcounter{page}{2}
%\pagestyle{plain}
%
%\begin{otherlanguage}{spanish}
%\begin{abstract}
%\phantomsection
%\addcontentsline{toc}{chapter}{Abstract}
This work presents a statistical approach to identify the
underlying roughness characteristics in synthetic aperture radar (SAR)
intensity data. The physical modeling of this kind of data allows the
use of the Gamma distribution in the presence of fully-developed
speckle, i.e., when there are infinitely many independent backscatterers
per resolution cell, and none dominates the return. Such areas are often
called ``homogeneous'' or ``textureless'' regions. The
\(\mathcal{G}_I^0\) distribution is also a widely accepted law for
heterogeneous and extremely heterogeneous regions, i.e., areas where the
fully-developed speckle hypotheses do not hold.
One issue involving the parametric space of $\mathcal{G}_I^0$ is the analytical 
infeasibility of testing homogeneity against heterogeneity using classical tests. 
As solutions to this problem, we propose three test
statistics to distinguish between homogeneous and inhomogeneous regions,
i.e., between Gamma and \(\mathcal{G}_I^0\) distributed data, both with
a known number of looks. The first test statistic uses a bootstrapped
non-parametric estimator of Shannon entropy, providing an
assessment in uncertain distributional assumptions. The second test uses
the classical coefficient of variation (CV). The third test uses an
alternative form of estimating the CV based on the ratio of the mean
absolute deviation from the median to the median. We apply our test
statistic to create maps of \(p\)-values for the homogeneity hypothesis.
Finally, we show that our proposal, the entropy-based test, outperforms
existing methods, such as the classical CV and its alternative variant,
in identifying heterogeneity when applied to both simulated and actual
data.
%\end{abstract}

\par
\noindent \textbf{Keywords:} SAR; heterogeneity; entropy; coefficient of variation; hypothesis tests



%\end{otherlanguage}
%\addcontentsline{toc}{chapter}{Abstract}
%\addcontentsline{toc}{chapter}{\protect\numberline{\abstrname}}
\clearpage

%\addcontentsline{toc}{section}{abstract}
%\abstract
\newpage
Este trabalho apresenta uma abordagem estatística para detectar heterogeneidade em dados de
intensidade de radar de abertura sintética (SAR), utilizando métodos baseados em entropia.
Na análise de dados SAR, uma interpretação precisa do terreno depende fundamentalmente da
distinção entre dois regimes principais: regiões homogêneas, em que o speckle está
totalmente desenvolvido, os retornos SAR são representados pela distribuição Gamma, e áreas heterogêneas requerem distribuições mais flexíveis para descrever
o espalhamento complexo, geralmente representado pela distribuição
$\mathcal{G}_I^0$.
Embora essa discriminação seja essencial para aplicações de sensoriamento remoto, 
testes paramétricos clássicos geralmente não são adequados para essa tarefa devido a 
limitações analíticas e numéricas.
Para superar esses desafios, propomos três estatísticas de teste para detectar heterogeneidade
em imagens SAR, com base nas entropias de Shannon, Rényi e Tsallis. Os testes associados utilizam
estimadores não paramétricos de entropia construídos a partir de espaçamentos amostrais,
evitando suposições explícitas sobre a distribuição subjacente. Para aumentar a precisão dos testes,
especialmente em amostras pequenas, incorporamos um procedimento de correção de viés via bootstrap,
que melhora a estabilidade dos estimadores, reduz o viés e o erro quadrático médio.
Os testes propostos são avaliados por meio de simulações de Monte Carlo,
tendo como critério de avaliação seu tamanho e poder sob diferentes condições de speckle e textura.
Os resultados mostram que os testes baseados nas entropias de Rényi e Tsallis superam a versão baseada
na entropia de Shannon, detectando variações de textura mais sutis e mantendo maior confiabilidade na
identificação de regiões verdadeiramente homogêneas.
Por fim, a metodologia é aplicada tanto a dados simulados como a dados SAR reais.
A análise é realizada com janelas deslizantes, gerando mapas de valores-$p$ que permitem a
avaliação visual e quantitativa da heterogeneidade espacial. O teste baseado na entropia de
Rényi mostra desempenho superior na identificação de padrões de rugosidade em pequena escala,
enquanto o teste baseado em Tsallis é melhor na detecção de regiões homogêneas. Em conjunto,
essas ferramentas baseadas em entropia oferecem uma estrutura robusta, interpretável e não
supervisionada para a detecção de heterogeneidade em dados SAR.

\vspace{1em}
\par
\noindent \textbf{Palavras-chave:} entropia;  heterogeneidade; distribuição gama; teste de hipótese; bootstrap.
%\addcontentsline{toc}{chapter}{Resumo}
%\addcontentsline{toc}{chapter}{\protect\numberline{\resumoname}}
\clearpage
\newpage
% Palavras-chave do resumo em Inglês
%\begin{keywords}
%DIGITE AS PALAVRAS-CHAVE AQUI
%\end{keywords}

% Sumário
% Comente para ocultar
\tableofcontents

%\cleardoublepage
%\setcounter{page}{-1}
%\newpage
% Lista de figuras
% Comente para ocultar
%\listoffigures
%\newpage
% Lista de tabelas
% Comente para ocultar
%\listoftables
\newpage



%%
%% Parte textual
%%
\mainmatter
%\pagestyle{thesis} % Return the page headers back to the "thesis" style
% É aconselhável criar cada capítulo em um arquivo à parte, digamos
% "capitulo1.tex", "capitulo2.tex", ... "capituloN.tex" e depois
% incluí-los com:
% \include{capitulo1}
% \include{capitulo2}
% ...
% \include{capituloN}
%\cleardoubleemptypage
%\newpage
%\pagenumbering{arabic}


%\setcounter{page}{1}
\chapter{Introduction}\label{chp:int}
%------------------------------------
%	INTRO INTRO
%------------------------------------


The technology of Synthetic Aperture Radar (SAR) imaging operates on the comprehensive aperture principle to produce high-resolution SAR images.
In contrast to conventional optical remote sensing images, SAR images remain unaffected by external factors, making them suitable for identifying ground targets in diverse weather conditions and expansive areas.  
Consequently, SAR imagery has become essential for environmental monitoring, disaster management, agriculture, topography mapping, geological exploration, maritime navigation, defense, and climatology~\citep{Moreira2013, iglesias2013atmospheric, Mu2019}.  
However, the effective use of SAR
data depends on a thorough understanding of its statistical properties
because it is corrupted by speckle. This noise-like interference effect
is inherent in SAR data due to the coherent nature of the imaging
process~\citep{Argenti2013}.

Speckle in intensity format is non-Gaussian. 
Thus, SAR data require reliable statistical models for accurate processing. 
The \(\mathcal{G}^0\) distribution, which is suitable for SAR data, includes
the Gamma law as the limiting case for fully-developed
speckle~\citep{Ferreira2020} and provides flexibility with fewer
parameters for analysis.

When deciding which model is the best, practitioners face a problem. 
On the one hand, if they opt for the Gamma law when the data come from the
\(\mathcal{G}^0\) distribution, they lose all the information about the
number of scatterers, which is revealed by one of the parameters of the
latter model~\citep{Yue2021}. 
On the other hand, if they apply the \(\mathcal{G}^0\) distribution under fully developed speckle, maximum
likelihood estimation is tricky: bias increases making estimation
unreliable~\citep{VasconcellosFrerySilva:CompStat}, and the likelihood is
flat, so numerical optimization may not
converge~\citep{FreryCribariSouza:JASP:04}. 

Our work aims to improve the identification of potential roughness
features in SAR intensity data. 
Physical modeling of SAR data allows the
use of the Gamma distribution in the presence of fully-developed
speckle, where an infinite number of independent backscatterers per
resolution unit is assumed, commonly referred to as homogeneous regions.

In this context, we present a set of three novel test statistics that
aim to distinguish between homogeneous and non-homogeneous returns,
particularly between gamma and \(\mathcal{G}^0\) distributed data,
assuming the number of looks is known.
 We use properties such as entropy and coefficient of variation.

Entropy is a fundamental concept in information theory with far-reaching
applications in pattern recognition, statistical physics, image
processing, edge detection and SAR image
analysis~\citep{Presse2013,MohammadDjafari2015,Avval2021, Nascimento2014,Nascimento2019}.
\citet{Shannon1948} this concept for a random variable to measure information and uncertainty. 
Shannon entropy is a crucial descriptive parameter in statistics, especially for evaluating data
dispersion and performing tests for normality, exponentiality and
uniformity~\citep{Wieczorkowski1999,Zamanzade2012}. 
Entropy estimation is challenging, especially when the model is unknown. In these cases,
non-parametric methods are used. Spacing methods have been discussed as
a non-parametric approach in Refs.~\citep{AlizadehNoughabi2010,Subhash2021}. This strategy is flexible
and robust because it does not enforce a model or parametric
constraints.

The coefficient of variation (CV), introduced in 1896 by~\citet{Pearson1896}, is a relative dispersion measure widely used
in various fields of applied statistics, including sampling,
biostatistics, medical and biological research, climatology and other
fields~\citep{hendricks1936sampling,Tian2005,SubrahmanyaNairy2003,Chankham2024}.
It facilitates the comparison of variability between different
populations and is particularly valuable for relating variables with
different units. This is because when the primary purpose is to compare
the variations of several variables, the standard deviation can only
serve as an adequate measure of variation if all variables are expressed
in the same unit of measurement and have identical means. If these
conditions are not met, then the CV is the relative measure that is
usually used in real applications. The variable with the highest CV
value has the largest relative dispersion around the mean
value~\citep{Banik2011}. The coefficient of variation is the primary
measure of heterogeneity in SAR data~\citep{Ulaby1986,Touzi1988}. We
study two ways of estimating the coefficient of variation.

The other parameter we study is the Shannon entropy. Different roughness
levels materialized as models for SAR data, have different entropy
values, but this fundamental quantity can also be estimated in a
model-agnostic way. We exploit this property and design a
bootstrap-improved non-parametric estimator for the Shannon entropy.

We devise test statistics based on these three estimators: the classical
coefficient of variation, a robust version, and the Shannon entropy
estimator. We apply these test statistics to generate maps of evidence
of homogeneity that reveal different types of targets in the SAR data.
We show that our proposed method is superior to existing approaches with
simulated data and SAR images.




%---------------------------------------------------------------------------------------------------------------------

\section{Objectives}

The aim of this work is the identification of roughness features in SAR intensity data by developing novel test statistics for distinguishing between homogeneous and heterogeneous domains. To achieve this goal, we propose the following procedure:

\begin{itemize}
    \item Select the optimal non-parametric estimator of entropy, refined through bootstrap techniques, aiming to reduce bias and mean square error.
    \item Propose three test statistics: the first based on the selected non-parametric estimator of entropy; the second utilizing the classical coefficient of variation; and the third using a robust approach of CV variant.
    \item Conduct computational experiments to evaluate the performance of the proposed test statistics across various simulated data scenarios.
    \item Apply the test statistics to SAR data to assess their effectiveness in identifying roughness features.
\end{itemize}

\subsection{Submitted Articles}

The results obtained in this work were submitted for publication as follows:
\begin{itemize}
    \item  MIGARS 2024 proceedings in the IEEE Xplore Digital Library®, titled "Identifying Departures from the Fully Developed Speckle Hypothesis in Intensity SAR Data with Non-Parametric Estimation of the Entropy".
    \item  Remote Sensing journal from MDPI, titled "Identifying Heterogeneity in SAR Data with New Test Statistics".
\end{itemize}
The articles were written in Rmarkdown and are fully reproducible. 
The code and data are accessible at 
\href{https://github.com/rjaneth/identifying-heterogeneity-in-sar-data-with-new-test-statistics}{\textcolor[rgb]{0,0,1}{Repository Link}}


\section{Manuscript organization}\label{sec:research_questions}

% colocar para citar los capitulos
\hypersetup{linkcolor=blue}
This document is organized as follows: 
Chapter~\ref{chp:background} provides a background on remote sensing and SAR images. 
Chapter~\ref{chp:methods} discusses methodological aspects, including statistical modeling, entropy estimation for intensity SAR data, and hypothesis testing. 
Chapter~\ref{chp:results} presents the main results obtained in our study with both simulated and actual data. Finally,
In Chapter~\ref{chp:conclusions}, the conclusions and future work are presented.
%Chapter ~\ref{chp:models} explains ...
%------------------------------------
%	HAZARD: OVERVIEW
%------------------------------------


%\section{Computing platforms}
%This  project required a vast amount of computing, preprocessing, analysis and plotting.

%------------------------------------
%	TODO STUFF
%------------------------------------
% \section{Additional sources to integrate in this chapter}
% \begin{itemize}
% \item check all TODOs
% \item read again \url{http://ec.europa.eu/environment/water/flood_risk/flood_atlas/pdf/handbook_goodpractice.pdf}
% \item include citation EU Floods Directive (2007/60/EC) , \cite{Mysiak2013, EUFD2007}
% \item section to talk about uncertainties? \citet{alfieri2014} has a good section about it
% \item Alps are the water tower of the Po plain, and for this reason I might want to give them a closer look. I could cite one of the many kotlarski works, or e.g. \citet[][]{Gobiet2014}. View the 'Alps' category in my Mendeley, it contains about 30 works on the subject
% \end{itemize}
\chapter{SAR Polarimetry}\label{chp:obs}%\Cref{sec:pr_obs}
%The main observational data used within this thesis are of four different kinds: precipitation, discharge, 

%------------------------------------
%	PRECIPITATION OBS
%------------------------------------

\section{Electromagnetic radiation}
\section{SAR system characteristics} \label{sec:pr_obs}
%\subsection*{Phase and Polarization}
%Precipitation is probably the most difficult of all atmospheric climate variables to measure reliably, due to the huge spatio-temporal variability (especially in summer) and to the physical difficulty of setting up and maintaining a dense network of high-maintenance sensors. In our multi-model approach
\section{Polarimetry}

%\subsection*{Generating Polarimetric Parameters in SNAP}
	\begin{figure}[H]
    \centering
    \includegraphics[width=0.4\textwidth]{figures/snap.PNG}
    \decoRule
    \caption[Generating Polarimetric Parameters in SNAP]{Generating Polarimetric Parameters in SNAP.}
    \label{fig:undercatch}
\end{figure}

	
	%\begin{wrapfigure}{l}{0.25\textwidth}
%\includegraphics[width=0.9\linewidth]{figures/PolSAR.jpg}
%\caption{Ttulo 1}
%\label{F:figuranotexto}
%\end{wrapfigure}


\subsection*{Example math equations}
%-----------------------------------------------------------
%\begin{align*}
%L(\alpha,  \lambda|  \bm{x})&=\prod_{i=1}^n{f(x_i|\alpha, \lambda )}\\
                                 %&= \prod_{i=1}^n\left[\frac{\lambda^{\alpha}}{\Gamma(\alpha)}\,x_i^{\alpha-1}\exp\left\{-\lambda x_i\right\}\right]\\
%&=\left(\frac{\lambda^{\alpha}}{\Gamma(\alpha)}\right)^n\prod_{i=1}^nx_i^{\alpha-1}\times \exp\left\{-\left(\lambda\sum_{i=1}^n x_i \right)\right\}
%\end{align*}
%%----------------------------------------------------------
\begin{align}
\label{eq:em}
\ell(\alpha, \lambda| \bm{x}) &=\log\left[\left(\frac{\lambda^{\alpha}}{\Gamma(\alpha)}\right)^n\prod_{i=1}^nx_i^{\alpha-1}\times \exp\left\{-\left(\lambda\sum_{i=1}^n x_i \right)\right\}\right]\nonumber\\
&=\log\left(\frac{\lambda^{\alpha}}{\Gamma(\alpha)}\right)^n+\log\left[\prod_{i=1}^nx_i^{\alpha-1}\right]-\lambda\sum_{i=1}^nx_i\nonumber\\
&=n\log\left(\frac{\lambda^{\alpha}}{\Gamma(\alpha)}\right)+(\alpha-1)\log\prod_{i=1}^nx_i-\lambda\sum_{i=1}^nx_i\nonumber\\
&=n\log(\lambda^{\alpha})-n\log(\Gamma(\alpha))+(\alpha-1)\sum_{i=1}^n\log(x_i)-\lambda\sum_{i=1}^nx_i\nonumber\\
&=n\alpha\log \lambda -n\log(\Gamma(\alpha)) +(\alpha-1)\sum_{i=1}^n\log(x_i)-\lambda\sum_{i=1}^nx_i.
	\end{align}
%----------------------------------------------------------
\begin{figure}
    \centering
    \includegraphics[width=0.5\textwidth]{figures/PolSAR.jpg}
    \decoRule
    \caption[Example plot]{Example plot.}
    \label{fig:undercatch}
\end{figure}

%-------------------------------------------------------

%\begin{table}[h]
%\centering
%\caption{\label{tab1}Comparação de tempos. }
%\begin{tabular}{l c c }
%\toprule[1pt] \textbf{Amostra (N)} \ \ \  \ \ \  & \textbf{Tempo de Execução (C)}\ \ \ \ \ \ \  &\textbf{Tempo de Execução (Ox)} \\\midrule[0.5pt]
%$20$ & 7.6810 s  &  5.01 s \\
%$30$ & 9.6260 s     & 5.34 s \\
%$50$ &  10.0310 s & 5.99 s \\
%$100$ & 15.5420 s & 8.07 s \\
%$200$ & 20.5860 s & 9.64 s \\
%$500$ & 29.0890  s & 15.12 s \\
%\bottomrule[1pt]
%\end{tabular}
%\caption[Precipitation datasets]{List of datasets used in the analysis of daily precipitation uncertainty carried over i}\label{tab:1}
%\end{table}
%------------------------------------------------------


%\begin{table}[]
%\centering
%\begin{tabular}{@{}llll@{}}
%\toprule
%Dataset name & Period & Spatial res. & Data source \\ \midrule
%E-OBS        & 2000--2016 & \ang{0.25}                            & Station data \\
%EURO4M-APGD  & 2000--2008 & \SI{5}{\kilo\metre}                   & Station data \\
%HMR          & 2000--2013 & \SI{5.5}{\kilo\metre}                 & Reanalysis \\
%ARCIS        & 2000--2015 & \textasciitilde{} \SI{5}{\kilo\metre} & Station data \\
%CHIRPS       & 2000--2016 & \ang{0.05}                            & Station data + satellite \\
%CPC          & 2000--2016 & \ang{0.5}                             & Station data \\
%CMORPH       & 2000--2016 & \ang{0.25}                            & Satellite \\
%PERSIANN-CDR & 2000--2016 & \ang{0.25}                            & Satellite \\ \bottomrule
%\end{tabular}
%\caption[ Italy]{List of datasets used in the analysis of daily precipitation uncertainty  references.}\label{tab:2}
%\end{table}




 
%\chapter{Data}\label{chp:itaobs}

\chapter{Methodology} \label{chp:models}

As introduced in Chapter ~\ref{chp:int}, several methods have also been proposed to...\\
This Chapter is organized as follows.


\chapter{Results}\label{chp:results}
This chapter will discuss the results obtained during the development of this project.\\
As described in the previous chapter... 
\chapter{Conclusions and future perspectives}\label{chp:conclusions}

As part of the project, we...\\

The work presented in this thesis..\\

Despite these limitations, we believe that the approach ...\\

In conclusion, ....

%\include{todo}

%%
%% Parte pós-textual
%%
\backmatter

% Apêndices
% Comente se não houver apêndices
\newpage


%\appendix

%------------------------------------------------------------
%activar luego
    %\refstepcounter{chapter}
    %\chapter*{\appendixname\enskip\thechapter}
    %\addcontentsline{toc}{chapter}{\appendixname\enskip\thechapter}
    %
    %\section{Section in Appendix A}
	%
		
%
    \refstepcounter{chapter}
    %\chapter*{\appendixname\enskip\thechapter}
    %\addcontentsline{toc}{chapter}{\appendixname\enskip\thechapter}

    %\section{ Appendix A}
			%\include{appendices/appendixB}
		%----------------------------------------
\cleardoublepage
%\newpage
% \phantomsection
%------------------------------------
%	ABBREVIATIONS
%------------------------------------

%% --- SOLO afecta a esta tabla ---
{%
\setlength{\extrarowheight}{0.5pt}   % +3pt por fila (prueba 2–4pt)
\renewcommand{\arraystretch}{1.18} % factor global (1.0=normal)

% Si quieres que las definiciones envuelvan línea, usa p{...}:
% \begin{abbreviations}{@{}p{3cm}p{11cm}@{}}
\begin{abbreviations}{@{}ll@{}}    % o deja "ll" si no necesitas ajuste de ancho
\textbf{AUC}    & Area Under the ROC Curve \\
\textbf{ASF}    & Alaska Satellite Facility \\
\textbf{CDF}    & Cumulative distribution function \\
\textbf{ENL}    & Equivalent number of looks \\
\textbf{ESA}    & European Space Agency \\
\textbf{PDF}    & Probability density function \\
\textbf{ROI}    & Region of interest \\
\textbf{ROC}    & Receiver Operating Characteristic \\
\textbf{SAR}    & Synthetic Aperture Radar \\
\textbf{SNAP}   & Sentinel Application Platform \\
\textbf{UAVSAR} & Uninhabited Aerial Vehicle Synthetic Aperture Radar \\
\end{abbreviations}
}% --- fin ámbito local ---


% \begin{abbreviations}{ll}
% \textbf{AUC}   & Area Under the ROC Curve \\
% \textbf{ASF}   & Alaska Satellite Facility \\
% \textbf{CDF}   & Cumulative distribution function \\
% \textbf{ENL}   & Equivalent number of looks \\
% \textbf{ESA}   & European Space Agency \\
% \textbf{PDF}   & Probability density function \\
% \textbf{ROI}   & Region of interest \\
% \textbf{ROC}   & Receiver Operating Characteristic \\
% \textbf{SAR}   & Synthetic Aperture Radar \\
% \textbf{SNAP}  & Sentinel Application Platform \\
% \textbf{UAVSAR} & Uninhabited Aerial Vehicle Synthetic Aperture Radar \\
% \end{abbreviations}

% \begin{abbreviations}{ll} % Include a list of abbreviations (a table of two columns)
% 
% \textbf{SAR} & Synthetic Aperture Radar\\
% \textbf{UAVSAR} & Uninhabited Aerial Vehicle Synthetic Aperture Radar\\
% \textbf{PDF} & Probability density function\\
% \textbf{CDF} & Cumulative distribution function\\
% \textbf{ASF} & Alaska Satellite Facility\\
% \textbf{ESA} & Agencia Espacial Europea\\
% \textbf{ASF} & Alaska Satellite Facility\\
% \textbf{ENL} & Equivalent Number of Looks\\
% \textbf{SNAP} & Sentinel Application Platform\\
% \textbf{ROI} & Region of Interest\\ 
% \textbf{ROC} & Receiver Operating Characteristic \\
% \textbf{AUC} & Area Under the ROC Curve
% \end{abbreviations}


%------------------------------------
{
  \hypersetup{linkcolor=black}% para colocar color negro en la tabla de listas y figuras
  

\newpage
\addcontentsline{toc}{chapter}{\listfigurename}
\listoffigures
\cleardoublepage
\newpage
\addcontentsline{toc}{chapter}{\listtablename}
\listoftables

}
% É aconselhável criar cada apêndice em um arquivo à parte, digamos
% "apendice1.tex", "apendice.tex", ... "apendiceM.tex" e depois
% incluí-los com:
% \include{apendice1}
% \include{apendice2}
% ...
% \include{apendiceM}
\newpage
\cleardoublepage
%------------------------------------
%	BIBLIOGRAPHY
%------------------------------------

%\bibliographystyle{IEEEtran}
%\bibliography{mendeley_v2}
%\nocite{*}
\printbibliography[heading=bibintoc]

%\setlength{\parskip}{3.0pt}
% Bibliografia
% É aconselhável utilizar o BibTeX a partir de um arquivo, digamos "biblio.bib".
% Para ajuda na criação do arquivo .bib e utilização do BibTeX, recorra ao
% BibTeXpress em www.cin.ufpe.br/~paguso/bibtexpress
%\nocite{*}
%\bibliographystyle{plain}
%\bibliography{biblio}

% Cólofon
% Descomente para incluir uma pequena nota com referência à UFPEThesis
%\colophon

%% Fim do documento
\end{document}
