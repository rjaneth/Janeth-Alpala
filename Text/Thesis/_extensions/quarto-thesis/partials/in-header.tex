%\usepackage[utf8]{inputenc}
%\usepackage[T1]{fontenc}
%\usepackage{mathpazo}
%\usepackage[backend=bibtex, style=authoryear, natbib=true]{biblatex}

\usepackage[autostyle=true]{csquotes}

\usepackage{pdfpages}
\usepackage{etoolbox} % utilidades

% ---- Paquetes extra tuyos ----
\usepackage{bbm}
\usepackage{mathrsfs}
\usepackage{url}
\usepackage{color}
\usepackage{mathtools}
\usepackage{array}
\usepackage{multirow}
\usepackage{wrapfig}
\usepackage{colortbl}
\usepackage{pdflscape}
\usepackage{xcolor}
\usepackage{float}
\usepackage{textcase} % para \MakeTextUppercase (seguro en ToC)

%----------------------------------------------------------------------------------------
% MÁRGENES
%----------------------------------------------------------------------------------------
\geometry{
  headheight=4ex,
  includehead,
  includefoot
}
\raggedbottom

%----------------------------------------------------------------------------------------
% METADATOS PDF
%----------------------------------------------------------------------------------------
\AtBeginDocument{
  \hypersetup{pdftitle=\ttitle}
  \hypersetup{pdfauthor=\authorname}
  \hypersetup{pdfkeywords=\keywordnames}
}

% ======================================================================================
%  ÍNDICE (ToC) ALINEADO Y FORMATO SEGÚN NORMAS
% ======================================================================================
\usepackage{tocloft}

% Líderes con puntos
\renewcommand{\cftdot}{.}
\renewcommand{\cftchapleader}{\cftdotfill{\cftdotsep}}
\renewcommand{\cftsecleader}{\cftdotfill{\cftdotsep}}
\renewcommand{\cftsubsecleader}{\cftdotfill{\cftdotsep}}

% Columna de numeración y comienzo del texto alineados por nivel
% {indent}{numwidth}  ← Ajusta 3.0em/4.0em/5.0em si lo necesitas
% MISMA COLUMNA DE TEXTO PARA TODOS LOS NIVELES
% Ajusta 5.0em–5.5em según la longitud máxima de tu numeración (p.ej. 9.9.9).
\cftsetindents{chapter}{0em}{5.0em}
\cftsetindents{section}{0em}{5.0em}
\cftsetindents{subsection}{0em}{5.0em}

\setlength{\cftbeforechapskip}{0.09em}
\setlength{\cftbeforesecskip}{0.08pt}       % antes de cada sección
\setlength{\cftbeforesubsecskip}{0.08pt}    % antes de cada subsección
%\setlength{\cftparskip}{0pt}             % espacio entre entradas consecutivas

% Fuentes en la ToC
% Capítulos: MAYÚSCULA + negrita
% Fuentes en la ToC
\renewcommand{\cftchapfont}{\bfseries\MakeTextUppercase}
\renewcommand{\cftchappagefont}{\bfseries}
\renewcommand{\cftsecfont}{\MakeTextUppercase}
\renewcommand{\cftsubsecfont}{\bfseries}
\renewcommand{\cftsubsecpagefont}{\bfseries} % quítalo si no quieres negrita en el número de página


% =======================
%  LISTAS (LoF / LoT)
%  "Figure 1 – Título", "Table 1 – Título"
% =======================
% Prefijos antes del número:
% \renewcommand{\cftfigpresnum}{Figure~}
% \renewcommand{\cfttabpresnum}{Table~}
% % Guion (con espacio) después del número:
% \renewcommand{\cftfigaftersnum}{\enspace--\enspace}
% \renewcommand{\cfttabaftersnum}{\enspace--\enspace}
% % --- Ensanchar columna "Figure N" / "Table N" en LOF/LOT ---
% \setlength{\cftfignumwidth}{6.0em}   % prueba 5.5–6.5em según lo largo de tus números
% \setlength{\cfttabnumwidth}{6.0em}

% LISTAS (LoF / LoT): “Figure 2.1 –” y “Table 1 –” en NEGRITA
% Nota: lo que quede entre presnum y aftersnum hereda \bfseries,
% así que el número también va en negrita; luego volvemos a \normalfont
% \renewcommand{\cftfigpresnum}{\bfseries \small Figure~}
% \renewcommand{\cftfigaftersnum}{\enspace--\enspace}
% 
% \renewcommand{\cfttabpresnum}{\bfseries \small Table~}
% \renewcommand{\cfttabaftersnum}{\enspace--\enspace}
% 
% 
% % Si necesitas ajustar el ancho reservado para “Figure 99 –”
% \setlength{\cftfignumwidth}{6.0em} % prueba 5.5–6.5em
% \setlength{\cfttabnumwidth}{6.0em}

% LISTAS (LoF / LoT): “Figure N – …” y “Table N – …” en negrita
\renewcommand{\cftfigpresnum}{\bfseries \small Figure~}
\renewcommand{\cftfigaftersnum}{\enspace--\enspace}
\renewcommand{\cfttabpresnum}{\bfseries \small Table~}
\renewcommand{\cfttabaftersnum}{\enspace--\enspace}

% Numeración global (sin prefijo de capítulo)
\makeatletter
\@removefromreset{figure}{chapter}
\@removefromreset{table}{chapter}
\makeatother
\renewcommand{\thefigure}{\arabic{figure}}
\renewcommand{\thetable}{\arabic{table}}

% Ajusta el ancho reservado para “Figure 99 –”
\setlength{\cftfignumwidth}{5.5em}
\setlength{\cfttabnumwidth}{5.0em}


% --- Separador de rótulos en el CUERPO: "Figure 2.3 -- Título"
\usepackage{caption} % (si ya lo carga la clase, no pasa nada repetir)
\DeclareCaptionLabelSeparator{dash}{\enspace\textendash\enspace}
\captionsetup[figure]{labelsep=dash}
\captionsetup[table]{labelsep=dash}

% Reservar ancho para número + guion (ajusta 99 si tienes >99):
%\settowidth{\cftfignumwidth}{Figure 99\enspace--\enspace}
%\settowidth{\cfttabnumwidth}{Table 99\enspace--\enspace}


% \renewcommand{\cftchapfont}{\bfseries\MakeUppercase}
% \renewcommand{\cftchappagefont}{\bfseries}
% 
% % Secciones: MAYÚSCULA (sin negrita)
% \renewcommand{\cftchapfont}{\bfseries\MakeTextUppercase}
% \renewcommand{\cftsecfont}{\MakeTextUppercase}
% 
% % Subsecciones: negrita y “sentence case”
% \renewcommand{\cftsubsecfont}{\bfseries}
% \renewcommand{\cftsubsecpagefont}{\bfseries} % quita \bfseries si no quieres negrita en el número de página

% (OPCIONAL) Interlineado 1.5 solo en la ToC
% \usepackage{setspace}
% \let\oldtableofcontents\tableofcontents
% \renewcommand{\tableofcontents}{%
%   \begingroup\onehalfspacing\oldtableofcontents\endgroup}

% ======================================================================================
%  TÍTULOS EN EL CUERPO (Capítulos/Secciones/Subsecciones)
% ======================================================================================
\usepackage{titlesec}

% Capítulos: número + TÍTULO EN MAYÚSCULA y NEGRITA (misma línea)
\titleformat{\chapter}[hang]
  {\normalfont\bfseries}     % formato
  {\thechapter}         % etiqueta (número)
  {1em}                 % separación número–título
  {\MakeUppercase}      % <--- sin #1; titlesec le pasa el título

% (Opcional) espaciamiento antes/después del capítulo
%\titlespacing*{\chapter}{0pt}{*3}{*2}
%\titlespacing*{\chapter}{0pt}{*2}{*1.2} 
\titlespacing*{\chapter}{0pt}{-1.0\baselineskip}{1.0\baselineskip}
% Secciones: MAYÚSCULA, sin negrita
\titleformat{\section}
%{\normalsize\normalfont} 
  {\normalfont\normalsize}
  {\thesection}{1em}{\MakeUppercase}  % <--- sin #1

% Subsecciones: negrita, “sentence case”
\titleformat{\subsection}
  {\bfseries\normalsize}
  {\thesubsection}{1em}{}             % {} → muestra el título tal cual

% Si también quieres definir subsubsection, hazlo así:
% \titleformat{\subsubsection}
%   {\normalfont\normalsize}
%   {\thesubsubsection}{1em}{}         % sin cambios de mayúsculas

% ======== Capítulos SIN número (frontmatter) centrados y tamaño adecuado ========
% Esto afecta a ABSTRACT, RESUMO, CONTENTS, LIST OF FIGURES/TABLES, etc.
\titleformat{name=\chapter,numberless}[block]
  {\bfseries\normalsize\centering} % formato: negrita, grande y CENTRADO
  {}                          % sin etiqueta (no hay número)
  {0pt}                       % separación
  {}                          % sin transformación extra (Abstract ya lo pone mayúsculas en su entorno)

% ======================================================================================
%  CAPCIONES EN EL TEXTO: usar guion en lugar de ":"
%  (la clase ya carga 'caption'; solo ajustamos el separador)
% ======================================================================================

% ===== APÉNDICES: sin línea "Appendices" en el ToC
%       y con entradas "APPENDIX A – <TÍTULO>" alineadas =====
% --- Quitar cabecera/entrada automática "Appendices" en el ToC ---

% ===== Apéndices: rótulo en capítulo y formato ToC =====
\usepackage{etoolbox}
\newcommand{\AppendixWord}{APPENDIX}

\makeatletter
\pretocmd{\appendix}{%
  % Numeración por letras
  \renewcommand{\thechapter}{\Alph{chapter}}%
  % Cabecera del capítulo: "APPENDIX A – TÍTULO"
  \titleformat{\chapter}[hang]
    {\normalfont\bfseries}
    {\AppendixWord~\thechapter\enspace\textendash}
    {1em}{\MakeUppercase}%
  % Formato en el sumario (ToC)
  \renewcommand{\cftchappresnum}{APPENDIX\enspace}%
  \renewcommand{\cftchapaftersnum}{\enspace\textendash\enspace}%
  % (Opcional) Ocultar A.1, A.2... en el ToC a partir de aquí
  \addtocontents{toc}{\protect\setcounter{tocdepth}{0}}%
  % Alinear la columna de texto de capítulos en ToC
  \cftsetindents{chapter}{0em}{11.5em}%
}{}{}
\makeatother

% --- Subcaptions más pequeños (solo (a), (b), …) ---



% --- Captions arriba y separador con guion ---
\usepackage{caption}
\DeclareCaptionLabelSeparator{dash}{\enspace\textendash\enspace}

\captionsetup[figure]{position=top,labelsep=dash}
\captionsetup[table]{position=top,labelsep=dash}
\captionsetup[sub]{justification=centering,font=footnotesize}%footnotesize



% --- Macro para "Source:" centrado y en chico ---
\newcommand{\FigSource}[1]{\begin{center}\footnotesize\textbf{Source:}~#1\end{center}}
\newcommand{\TableSource}[1]{\FigSource{#1}} % mismo estilo para tablas


% Añádelo al FINAL del archivo, después de cualquier \captionsetup
\usepackage{etoolbox}
\AtEndEnvironment{figure}{\par\FigSource{The author (2025)}}
\AtEndEnvironment{table}{\par\TableSource{The author (2025)}}
\usepackage{indentfirst}


% --- Interlineado un poco mayor que 1.5 ---
\usepackage{setspace}       % inofensivo aunque la clase ya lo cargue
\setstretch{1.5}           % prueba 1.55–1.62 hasta quedar a gusto


% % ---- Ocultar la línea "Appendices" del sumario (ToC) ----
% \makeatletter
% \let\mdt@orig@l@chapter\l@chapter
% \renewcommand*\l@chapter[2]{%
%   % #1 = título en el ToC, #2 = número de página
%   \begingroup
%     % título estándar que pone KOMA para la página de apéndices
%     \edef\mdt@apxname{\appendixpagename}%
%     % si no existiera \appendixpagename, usa literal "Appendices"
%     \@ifundefined{appendixpagename}{\def\mdt@apxname{Appendices}}{}%
%     % ¿el título que llegó es exactamente ese? -> no lo imprimimos
%     \edef\mdt@this{#1}%
%     \ifnum\pdfstrcmp{\mdt@this}{\mdt@apxname}=0\relax
%       % saltar
%     \else
%       \mdt@orig@l@chapter{#1}{#2}%
%     \fi
%   \endgroup
% }
% \makeatother


% \usepackage{etoolbox}
% \makeatletter
% \pretocmd{\appendix}{%
%   % Desactiva comandos típicos de paquetes/clases que insertan esa línea
%   \let\appendixpage\relax
%   \let\appendixpagename\relax
%   \let\appendixtocname\relax
%   \let\addappheadtotoc\relax
% }{}{}
% \makeatother
% 
% % --- ToC: Prefijo "APPENDIX A – " y misma tabulación que capítulos ---
% \makeatletter
% \pretocmd{\appendix}{%
%   % Guardar definición original
%   \let\orig@numberline\numberline
%   % Hacer que el "número" del capítulo en ToC sea "APPENDIX <letra> – "
%   \renewcommand{\numberline}[1]{APPENDIX~##1\enspace\textendash\enspace}
%   % Alineación: mismos márgenes/columnas que los capítulos normales
%   \cftsetindents{chapter}{0em}{11.5em}% ajusta 11.0–12.0em si lo necesitas
% }{}{}
% % Restaurar al final del documento (por seguridad)
% \apptocmd{\enddocument}{\let\numberline\orig@numberline}{}{}
% \makeatother


% \newcommand{\AppendixWord}{APPENDIX}
% 
% \makeatletter
% \pretocmd{\appendix}{%
%   % Numeración A, B, ...
%   \renewcommand{\thechapter}{\Alph{chapter}}%
% 
%   % Título del capítulo en el cuerpo:
%   \titleformat{\chapter}[hang]
%     {\normalfont\bfseries}
%     {\AppendixWord~\thechapter\enspace\textendash}
%     {1em}{\MakeUppercase}%
% 
%   % Entrada del ToC: que el “APPENDIX A –” sea parte del texto (no una columna aparte)
%   \renewcommand{\cftchappresnum}{\AppendixWord\enspace}%
%   \renewcommand{\cftchapaftersnum}{\enspace\textendash\enspace}%
% }{}{}
% \makeatother
% \usepackage{etoolbox}
% \newcommand{\AppendixWord}{APPENDIX}
% 
% \makeatletter
% \pretocmd{\appendix}{%
%   \renewcommand{\thechapter}{\Alph{chapter}}%
%   \titleformat{\chapter}[hang]
%     {\normalfont\bfseries}
%     {\AppendixWord~\thechapter\enspace\textendash}
%     {1em}{\MakeUppercase}%
%   \renewcommand{\cftchappresnum}{APPENDIX\enspace}%
%   \renewcommand{\cftchapaftersnum}{\enspace\textendash\enspace}%
%   \cftsetindents{chapter}{0em}{11.5em}%
% }{}{}
% \makeatother

% \usepackage{etoolbox}
% \newcommand{\AppendixWord}{APPENDIX}
% 
% \makeatletter
% \pretocmd{\appendix}{%
%   \renewcommand{\thechapter}{\Alph{chapter}}%
%   \titleformat{\chapter}[hang]
%     {\normalfont\bfseries}
%     {\AppendixWord~\thechapter\enspace\textendash}
%     {1em}{\MakeUppercase}%
%   \renewcommand{\cftchappresnum}{APPENDIX\enspace}%
%   \renewcommand{\cftchapaftersnum}{\enspace\textendash\enspace}%
%   \cftsetindents{chapter}{0em}{11.5em}% ajusta 11.0–12.0em si hiciera falta
% }{}{}
% \makeatother

% ===== Apéndices en el ToC: sin página "APPENDICES" y con "APPENDIX A – <Título>" =====
% \usepackage{etoolbox}
% \newcommand{\AppendixWord}{APPENDIX}
% 
% \makeatletter
% \pretocmd{\appendix}{%
%   \renewcommand{\thechapter}{\Alph{chapter}}%
%   \titleformat{\chapter}[hang]
%     {\normalfont\bfseries}
%     {\AppendixWord~\thechapter\enspace\textendash}
%     {1em}{\MakeUppercase}%
%   \renewcommand{\cftchappresnum}{APPENDIX\enspace}%
%   \renewcommand{\cftchapaftersnum}{\enspace\textendash\enspace}%
%   \cftsetindents{chapter}{0em}{11.5em}% ajusta 11.0–12.0em si hiciera falta
% }{}{}
% \makeatother

% % ===== Apéndices: rotulado y ToC =====
% \usepackage{etoolbox}
% \newcommand{\AppendixWord}{APPENDIX}
% 
% % Cuando entremos al modo apéndice:
% \makeatletter
% \pretocmd{\appendix}{%
%   % Numeración A, B, C...
%   \renewcommand{\thechapter}{\Alph{chapter}}%
%   % Cabecera en el cuerpo: "APPENDIX A – <TÍTULO>"
%   \titleformat{\chapter}[hang]
%     {\normalfont\bfseries}
%     {\AppendixWord~\thechapter\enspace\textendash}
%     {1em}{\MakeUppercase}%
%   % ToC: prefijo/sufijo y ancho reservado para "APPENDIX A –"
%   \renewcommand{\cftchappresnum}{APPENDIX\enspace}%
%   \renewcommand{\cftchapaftersnum}{\enspace\textendash\enspace}%
%   \cftsetindents{chapter}{0em}{11.5em}%
%   % A partir de aquí, solo listar a nivel capítulo en el ToC
%   \addtocontents{toc}{\protect\setcounter{tocdepth}{0}}%
% }{}{}
% \makeatother
% 
% 

% 
% % ===== Apéndices en el ToC: sin "Appendices" y con "APPENDIX A – <Título>" =====
% \usepackage{etoolbox}
% \newcommand{\AppendixWord}{APPENDIX}
% 
% % Formato del ToC para capítulos cuando estemos en apéndices
% \newcommand{\AppendixTocOn}{%
%   % Prefijo/sufijo alrededor del número "A", "B", ...
%   \renewcommand{\cftchappresnum}{APPENDIX\enspace}%
%   \renewcommand{\cftchapaftersnum}{\enspace\textendash\enspace}%
%   % Ajusta este ancho si hace falta (10.5–12.5em)
%   \cftsetindents{chapter}{0em}{11.5em}%
% }
% 
% \makeatletter
% \pretocmd{\appendix}{%
%   % Numeración A, B, C...
%   \renewcommand{\thechapter}{\Alph{chapter}}%
%   % Cabecera en el cuerpo: "APPENDIX A – <TÍTULO>"
%   \titleformat{\chapter}[hang]
%     {\normalfont\bfseries}
%     {\AppendixWord~\thechapter\enspace\textendash}
%     {1em}{\MakeUppercase}%
%   % Formato en ToC
%   \AppendixTocOn
%   % Si estamos en KOMA, suprime la entrada "Appendices" del ToC
%   \@ifundefined{addparttocentry}{}{%
%     \let\oldaddparttocentry\addparttocentry
%     \renewcommand*\addparttocentry[2]{}%
%   }%
% }{}{}
% 
% % (Opcional) restaurar si estabas en KOMA
% \apptocmd{\enddocument}{%
%   \@ifundefined{oldaddparttocentry}{}{%
%     \let\addparttocentry\oldaddparttocentry
%   }%
% }{}{}
% \makeatother


% % ================== APÉNDICES: rotulado, guion y ToC ==================
% \usepackage{etoolbox}
% \newcommand{\AppendixWord}{APPENDIX}
% 
% % Prefijo/sufijo en el ToC para capítulos (aplica cuando estemos en apéndices)
% \newcommand{\AppendixTocOn}{%
%   \renewcommand{\cftchappresnum}{APPENDIX\enspace}%
%   \renewcommand{\cftchapaftersnum}{\enspace\textendash\enspace}%
% }
% \newcommand{\AppendixHideSecs}{%
%   \setcounter{secnumdepth}{0}%
%   \addtocontents{toc}{\protect\setcounter{tocdepth}{0}}%
% }
% 
% \makeatletter
% \pretocmd{\appendix}{%
%   % A, B, C... en la numeración de capítulo
%   \renewcommand{\thechapter}{\Alph{chapter}}%
%   % Cabecera en el cuerpo: "APPENDIX A – <TÍTULO>"
%   \titleformat{\chapter}[hang]
%     {\normalfont\bfseries}%
%     {\AppendixWord~\thechapter\enspace\textendash}%
%     {1em}{\MakeUppercase}%
%   % Activar formato del ToC y ocultar secciones en ToC
%   \AppendixTocOn
%   \AppendixHideSecs
% }{}{}
% \makeatother

% % --- Separador personalizado para leyendas: "Figure 1 – Título" ---
% \DeclareCaptionLabelSeparator{ufpeDash}{\enspace\textendash\enspace} % en dash con espacio
% \captionsetup[figure]{labelsep=ufpeDash}
% \captionsetup[table]{labelsep=ufpeDash}
% 
% % ================== APÉNDICES: rotulado, guion y ToC ==================
% \usepackage{etoolbox}
% \newcommand{\AppendixWord}{APPENDIX} % Palabra delante de la letra
% 
% \makeatletter
% \pretocmd{\appendix}{%
%   % Capítulos con letras A, B, ...
%   \renewcommand{\thechapter}{\Alph{chapter}}%
%   % Cabecera en el cuerpo: "APPENDIX A – <TÍTULO>"
%   \titleformat{\chapter}[hang]
%     {\normalfont\bfseries}%
%     {\AppendixWord~\thechapter\enspace\textendash}%
%     {1em}{\MakeUppercase}%
%   % === ToC ===
%   % Prefijo "APPENDIX " antes de la letra en el ToC:
%   \renewcommand{\cftchappresnum}{APPENDIX\enspace}
%   % Guion entre número y título en el ToC:
%   \renewcommand{\cftchapaftersnum}{\enspace\textendash\enspace}%
%   % Ocultar numeración y entradas de secciones dentro de apéndices:
%   \setcounter{secnumdepth}{0}%
%   \addtocontents{toc}{\protect\setcounter{tocdepth}{0}}%
% }{}{}
% \makeatother


% % ================== APÉNDICES: rotulado, guion y ToC ==================
% \usepackage{etoolbox}
% \newcommand{\AppendixWord}{APPENDIX} % Palabra delante de la letra
% 
% \makeatletter
% \pretocmd{\appendix}{%
%   % A, B, C... como número de capítulo
%   \renewcommand{\thechapter}{\Alph{chapter}}%
%   % Cabecera de capítulo: "APPENDIX A – <TÍTULO>"
%   \titleformat{\chapter}[hang]
%     {\normalfont\bfseries}%
%     {\AppendixWord~\thechapter\enspace\textendash}%
%     {1em}{\MakeUppercase}%
%   % En el ToC, reserva caja de número y añade guion: "APPENDIX A – TÍTULO"
%   \renewcommand{\cftchapaftersnum}{\enspace\textendash\enspace}%
%   % --- Reglas específicas pedidas por la uni ---
%   \setcounter{secnumdepth}{0}% NO numerar secciones dentro de apéndices (sin A.1)
%   \addtocontents{toc}{\protect\setcounter{tocdepth}{0}}% Ocultar secciones en ToC
% }{}{}
% \makeatother


