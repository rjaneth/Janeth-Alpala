\chapter{Conclusions and future perspectives}\label{chp:conclusions}

This work provides a practical and theoretical answer to the
following physical question: How to detect heterogeneity in SAR images,
assuming that the SAR intensity follows the \(\Gamma_{\text{SAR}}\)
model. To this end, we proposed three novel hypothesis tests, one from
the Shannon entropy and two from the variation coefficient variants. The
performance of our proposals was evaluated using a Monte Carlo study.
The results showed that they were conservative in estimating the
probability of a type I error (false alarm rate) and the test power
(probability of detection), which increases with sample size. An
application to three recent SAR images was performed. The results showed
that the Shannon entropy-based test was more robust than the CV-based
tests. In addition, all tests could recognize images with different
textures and identify edges where the texture type changes.


\section*{Future perspectives}

For future research directions, there is a promising avenue in estimating the equivalent number of looks (ENL), a crucial parameter in the statistical modeling of multi-look synthetic aperture radar (SAR) imagery. This parameter serves as an indicator of heterogeneity and can significantly impact the accuracy of statistical analyses.

Additionally, we plan to explore the estimation of ENL in polarimetric SAR (PolSAR) data. This extension will involve:
\begin{itemize}
	\item Estimating the test on each of the three intensity channels of fully PolSAR data
	\item Analyzing the joint distribution
	\item Proposing techniques that generalize the test statistic  into the analysis of PolSAR data. 
		First, we can examine the entropy and CV of the distribution for the eigenvalues
		of the coherence matrix and then the Shannon entropy for the PolSAR matrix
			and trace-based versions for the CV of the PolSAR matrix.
\end{itemize}