\chapter{Introduction}\label{chp:int}
%------------------------------------
%	INTRO INTRO
%------------------------------------


The technology of Synthetic Aperture Radar (SAR) imaging operates on the comprehensive aperture principle to produce high-resolution SAR images.
In contrast to conventional optical remote sensing images, SAR images remain unaffected by external factors, making them suitable for identifying ground targets in diverse weather conditions and expansive areas~\citep{Mu2019}.  
Consequently, SAR imagery has become essential for environmental monitoring~\citep{Amitrano2021}, crop mapping~\citep{DingleRobertson2020}, ship target detection~\citep{Wang2019}, and climatology~\citep{Zhao2023}.  
However, the effective use of SAR
data depends on a thorough understanding of its statistical properties
because it is corrupted by speckle. This noise-like interference effect
is inherent in SAR data due to the coherent nature of the imaging
process~\citep{Argenti2013}.

Speckle in intensity format is non-Gaussian. 
Thus, SAR data require reliable statistical models for accurate processing. 
The \(\mathcal{G}^0\) distribution, which is suitable for SAR data, includes
the Gamma law as the limiting case for fully-developed
speckle~\citep{Ferreira2020} and provides flexibility with fewer
parameters for analysis.

When deciding which model is the best, practitioners face a problem. 
On the one hand, if they opt for the Gamma law when the data come from the
\(\mathcal{G}^0\) distribution, they lose all the information about the
number of scatterers, which is revealed by one of the parameters of the
latter model~\citep{Yue2021}. 
On the other hand, if they apply the \(\mathcal{G}^0\) distribution under fully developed speckle, maximum
likelihood estimation is tricky: bias increases making estimation
unreliable~\citep{VasconcellosFrerySilva:CompStat}, and the likelihood is
flat, so numerical optimization may not
converge~\citep{FreryCribariSouza:JASP:04}. 

Our work aims to improve the identification of potential roughness
features in SAR intensity data. 
Physical modeling of SAR data allows the
use of the Gamma distribution in the presence of fully-developed
speckle, where an infinite number of independent backscatterers per
resolution unit is assumed, commonly referred to as homogeneous regions.

Entropy is a fundamental concept in information theory with far-reaching
applications in pattern recognition~\citep{Avval2021}, statistical physics~\citep{Presse2013}, image
processing~\citep{MohammadDjafari2015}, edge detection~\citep{Nascimento2014} and SAR image
analysis~\citep{Nascimento2019}.
\citet{Shannon1948} introduced this concept for a random variable to measure information and uncertainty. 
Shannon entropy is a crucial descriptive parameter in statistics, especially for evaluating data
dispersion and performing tests for normality, exponentiality and
uniformity~\citep{Wieczorkowski1999,Zamanzade2012}. 

Entropy estimation is challenging, especially when the model is unknown. In these cases,
non-parametric methods are used. Spacing methods have been discussed as
a non-parametric approach in Refs.~\citep{AlizadehNoughabi2010,Subhash2021}. This strategy is flexible
and robust because it does not enforce a model or parametric
constraints. In this context, we introduce a novel approach: a bootstrap-improved non-parametric estimator for Shannon entropy.

The coefficient of variation (CV), introduced in 1896 by~\citet{Pearson1896}, is a relative dispersion measure widely used
in various fields of applied statistics, including sampling~\citep{hendricks1936sampling},
biostatistics~\citep{Tian2005}, medical and biological research~\citep{SubrahmanyaNairy2003}, climatology~\citep{Chankham2024} and other
fields. It facilitates the comparison of variability between different
populations and is particularly valuable for relating variables with
different units. The variable with the highest CV
value has the largest relative dispersion around the mean
value~\citep{Banik2011}. The coefficient of variation is the primary
measure of heterogeneity in SAR data~\citep{Ulaby1986,Touzi1988}. 
In our work, we employ both the classical CV and a robust alternative, which is based on the ratio of the mean absolute deviation from the median (MnAD) to the median.  These measures offer a more detailed view of data variability. 

The expected value, a critical component in both CV and the Shannon entropy, influences their interpretation. 
In CV calculations, the expected value serves as a reference point for assessing data dispersion relative to its average value, while in the Shannon entropy, the expected value also plays an important role, particularly within parametric distributions such as the Gamma or the \({G}_I^0\)  distribution. 
In these cases, the mean appears in the analytical entropy formula, indicating that the entropy of the distribution is influenced by the mean value of the data.


When applying these statistical concepts to SAR image analysis, we aim to discern between homogeneous and non-homogeneous regions, contributing to the improved interpretation and understanding of SAR data.  In our study, we devise test statistics based on these three estimators: the classical coefficient of variation, a robust version, and the Shannon entropy estimator. We apply these test statistics to generate maps of evidence of homogeneity that reveal different types of targets in the SAR data, assuming the number of looks is known. Likewise, we show that our proposed method is superior to existing approaches with simulated data and SAR images.




%---------------------------------------------------------------------------------------------------------------------

\section{Objectives}

The aim of this work is the identification of roughness features in SAR intensity data by developing novel test statistics for distinguishing between homogeneous and heterogeneous domains. 
To achieve this goal, we propose the following procedure:

\begin{itemize}
    \item Select the optimal non-parametric estimator of entropy, refined through bootstrap techniques, aiming to reduce bias and mean square error.
    \item Propose three test statistics: the first based on the selected non-parametric estimator of entropy; the second utilizing the classical coefficient of variation; and the third using a robust approach of CV variant.
    \item Conduct computational experiments to evaluate the performance of the proposed test statistics across various simulated data scenarios.
    \item Apply the test statistics to SAR data to assess their effectiveness in identifying roughness features.
\end{itemize}

\subsection*{Submitted Articles}

The results obtained in this work were submitted for publication as follows:
\begin{itemize}
    \item  MIGARS 2024 proceedings in the IEEE Xplore Digital Library®, titled "Identifying Departures from the Fully Developed Speckle Hypothesis in Intensity SAR Data with Non-Parametric Estimation of the Entropy".
    \item  Remote Sensing journal from MDPI, titled "Identifying Heterogeneity in SAR Data with New Test Statistics".
\end{itemize}
The articles were written in Rmarkdown and are fully reproducible. 
The code and data are accessible at 
\href{https://github.com/rjaneth/identifying-heterogeneity-in-sar-data-with-new-test-statistics}{\textcolor[rgb]{0,0,1}{Repository Link}}


\section{Manuscript organization}\label{sec:research_questions}

% colocar para citar los capitulos
\hypersetup{linkcolor=blue}
This document is organized as follows: 
Chapter~\ref{chp:background} provides a background on remote sensing and SAR images. 
Chapter~\ref{chp:methods} discusses methodological aspects, including statistical modeling, entropy estimation for intensity SAR data, and hypothesis testing. 
Chapter~\ref{chp:results} presents the main results obtained in our study with both simulated and actual data. Finally,
In Chapter~\ref{chp:conclusions}, the conclusions and future work are presented.

% \item section to talk about uncertainties? \citet{alfieri2014} has a good section about it
% \item Alps are the water tower of the Po plain, and for this reason I might want to give them a closer look. I could cite one of the many kotlarski works, or e.g. \citet[][]{Gobiet2014}. View the 'Alps' category in my Mendeley, it contains about 30 works on the subject
% \end{itemize}