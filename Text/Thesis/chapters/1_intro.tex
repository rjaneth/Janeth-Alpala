\chapter{Introduction}\label{chp:int}
%------------------------------------
%	INTRO INTRO
%------------------------------------


The technology of Synthetic Aperture Radar (SAR) imaging operates on the comprehensive aperture principle to produce high-resolution SAR images.
In contrast to conventional optical remote sensing images, SAR images remain unaffected by external factors, making them suitable for identifying ground targets in diverse weather conditions and expansive areas.  
Consequently, SAR imagery has become essential for environmental monitoring, disaster management, agriculture, topography mapping, geological exploration, maritime navigation, defense, and climatology~\citep{Moreira2013, iglesias2013atmospheric, Mu2019}.  
However, the effective use of SAR
data depends on a thorough understanding of its statistical properties
because it is corrupted by speckle. This noise-like interference effect
is inherent in SAR data due to the coherent nature of the imaging
process~\citep{Argenti2013}.

Speckle in intensity format is non-Gaussian. 
Thus, SAR data require reliable statistical models for accurate processing. 
The \(\mathcal{G}^0\) distribution, which is suitable for SAR data, includes
the Gamma law as the limiting case for fully-developed
speckle~\citep{Ferreira2020} and provides flexibility with fewer
parameters for analysis.

When deciding which model is the best, practitioners face a problem. 
On the one hand, if they opt for the Gamma law when the data come from the
\(\mathcal{G}^0\) distribution, they lose all the information about the
number of scatterers, which is revealed by one of the parameters of the
latter model~\citep{Yue2021}. 
On the other hand, if they apply the \(\mathcal{G}^0\) distribution under fully developed speckle, maximum
likelihood estimation is tricky: bias increases making estimation
unreliable~\citep{VasconcellosFrerySilva:CompStat}, and the likelihood is
flat, so numerical optimization may not
converge~\citep{FreryCribariSouza:JASP:04}. 

Our work aims to improve the identification of potential roughness
features in SAR intensity data. 
Physical modeling of SAR data allows the
use of the Gamma distribution in the presence of fully-developed
speckle, where an infinite number of independent backscatterers per
resolution unit is assumed, commonly referred to as homogeneous regions.

In this context, we present a set of three novel test statistics that
aim to distinguish between homogeneous and non-homogeneous returns,
particularly between gamma and \(\mathcal{G}^0\) distributed data,
assuming the number of looks is known.
 We use properties such as entropy and coefficient of variation.

Entropy is a fundamental concept in information theory with far-reaching
applications in pattern recognition, statistical physics, image
processing, edge detection and SAR image
analysis~\citep{Presse2013,MohammadDjafari2015,Avval2021, Nascimento2014,Nascimento2019}.
\citet{Shannon1948} this concept for a random variable to measure information and uncertainty. 
Shannon entropy is a crucial descriptive parameter in statistics, especially for evaluating data
dispersion and performing tests for normality, exponentiality and
uniformity~\citep{Wieczorkowski1999,Zamanzade2012}. 
Entropy estimation is challenging, especially when the model is unknown. In these cases,
non-parametric methods are used. Spacing methods have been discussed as
a non-parametric approach in Refs.~\citep{AlizadehNoughabi2010,Subhash2021}. This strategy is flexible
and robust because it does not enforce a model or parametric
constraints.

The coefficient of variation (CV), introduced in 1896 by~\citet{Pearson1896}, is a relative dispersion measure widely used
in various fields of applied statistics, including sampling,
biostatistics, medical and biological research, climatology and other
fields~\citep{hendricks1936sampling,Tian2005,SubrahmanyaNairy2003,Chankham2024}.
It facilitates the comparison of variability between different
populations and is particularly valuable for relating variables with
different units. This is because when the primary purpose is to compare
the variations of several variables, the standard deviation can only
serve as an adequate measure of variation if all variables are expressed
in the same unit of measurement and have identical means. If these
conditions are not met, then the CV is the relative measure that is
usually used in real applications. The variable with the highest CV
value has the largest relative dispersion around the mean
value~\citep{Banik2011}. The coefficient of variation is the primary
measure of heterogeneity in SAR data~\citep{Ulaby1986,Touzi1988}. We
study two ways of estimating the coefficient of variation.

The other parameter we study is the Shannon entropy. Different roughness
levels materialized as models for SAR data, have different entropy
values, but this fundamental quantity can also be estimated in a
model-agnostic way. We exploit this property and design a
bootstrap-improved non-parametric estimator for the Shannon entropy.

We devise test statistics based on these three estimators: the classical
coefficient of variation, a robust version, and the Shannon entropy
estimator. We apply these test statistics to generate maps of evidence
of homogeneity that reveal different types of targets in the SAR data.
We show that our proposed method is superior to existing approaches with
simulated data and SAR images.




%---------------------------------------------------------------------------------------------------------------------

\section{Objectives}

The aim of this work is the identification of roughness features in SAR intensity data by developing novel test statistics for distinguishing between homogeneous and heterogeneous domains. To achieve this goal, we propose the following procedure:

\begin{itemize}
    \item Select the optimal non-parametric estimator of entropy, refined through bootstrap techniques, aiming to reduce bias and mean square error.
    \item Propose three test statistics: the first based on the selected non-parametric estimator of entropy; the second utilizing the classical coefficient of variation; and the third using a robust approach of CV variant.
    \item Conduct computational experiments to evaluate the performance of the proposed test statistics across various simulated data scenarios.
    \item Apply the test statistics to SAR data to assess their effectiveness in identifying roughness features.
\end{itemize}

\subsection{Submitted Articles}

The results obtained in this work were submitted for publication as follows:
\begin{itemize}
    \item  MIGARS 2024 proceedings in the IEEE Xplore Digital Library®, titled "Identifying Departures from the Fully Developed Speckle Hypothesis in Intensity SAR Data with Non-Parametric Estimation of the Entropy".
    \item  Remote Sensing journal from MDPI, titled "Identifying Heterogeneity in SAR Data with New Test Statistics".
\end{itemize}
The articles were written in Rmarkdown and are fully reproducible. 
The code and data are accessible at 
\href{https://github.com/rjaneth/identifying-heterogeneity-in-sar-data-with-new-test-statistics}{\textcolor[rgb]{0,0,1}{Repository Link}}


\section{Manuscript organization}\label{sec:research_questions}

% colocar para citar los capitulos
\hypersetup{linkcolor=blue}
This document is organized as follows: 
Chapter~\ref{chp:background} provides a background on remote sensing and SAR images. 
Chapter~\ref{chp:methods} discusses methodological aspects, including statistical modeling, entropy estimation for intensity SAR data, and hypothesis testing. 
Chapter~\ref{chp:results} presents the main results obtained in our study with both simulated and actual data. Finally,
In Chapter~\ref{chp:conclusions}, the conclusions and future work are presented.
%Chapter ~\ref{chp:models} explains ...
%------------------------------------
%	HAZARD: OVERVIEW
%------------------------------------


%\section{Computing platforms}
%This  project required a vast amount of computing, preprocessing, analysis and plotting.

%------------------------------------
%	TODO STUFF
%------------------------------------
% \section{Additional sources to integrate in this chapter}
% \begin{itemize}
% \item check all TODOs
% \item read again \url{http://ec.europa.eu/environment/water/flood_risk/flood_atlas/pdf/handbook_goodpractice.pdf}
% \item include citation EU Floods Directive (2007/60/EC) , \cite{Mysiak2013, EUFD2007}
% \item section to talk about uncertainties? \citet{alfieri2014} has a good section about it
% \item Alps are the water tower of the Po plain, and for this reason I might want to give them a closer look. I could cite one of the many kotlarski works, or e.g. \citet[][]{Gobiet2014}. View the 'Alps' category in my Mendeley, it contains about 30 works on the subject
% \end{itemize}