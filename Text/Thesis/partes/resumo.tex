\begin{otherlanguage}{portuguese} 


\begin{resumo}
\phantomsection
\addcontentsline{toc}{chapter}{Resumo}

Este trabalho apresenta uma abordagem estatística para identificar as características de rugosidade subjacentes nos dados de intensidade do radar de abertura sintética (SAR). 
A modelagem física desse tipo de dados permite o uso da distribuição Gamma na presença de speckle totalmente desenvolvido, ou seja, quando há infinitos retroespalhadores independentes por celda de resolução e nenhum domina o retorno.
Essas áreas são frequentemente chamadas de regiões "homogêneas" ou "sem textura". 
A distribuição $\mathcal{G}_I^0$ também é uma lei amplamente aceita para regiões heterogêneas e extremamente heterogêneas, ou seja, áreas onde as hipóteses de speckle totalmente desenvolvidas não se aplicam. 
Propomos três estatísticas de teste para distinguir entre regiões homogêneas e inhomogêneas, ou seja, entre dados distribuídos gamma e $\mathcal{G}_I^0$, ambos com um número conhecido de looks. 
A primeira estatística de teste usa um estimador não paramétrico da entropia de Shannon, incorporando metodologias de bootstrap, fornecendo uma avaliação robusta em suposições onde a distribuição é desconhecida.
O segundo teste usa o coeficiente de variação clássico (CV). 
O terceiro teste usa uma forma alternativa de estimar o CV com base na razão da média do desvio absoluto em relação à mediana.
 Aplicamos nossa estatística de teste para criar mapas de $p$-valores para a hipótese de homogeneidade. 
Finalmente, mostramos que nossa proposta, o teste baseado em entropia, supera os métodos existentes, como o CV clássico e sua variante alternativa, na identificação de heterogeneidade quando aplicado a dados simulados e imagens SAR.


\end{resumo}
\end{otherlanguage}