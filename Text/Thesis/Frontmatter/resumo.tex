Este trabalho apresenta uma abordagem estatística para detectar heterogeneidade em dados de
intensidade de radar de abertura sintética (SAR), utilizando métodos baseados em entropia.
Na análise de dados SAR, uma interpretação precisa do terreno depende fundamentalmente da
distinção entre dois regimes principais: regiões homogêneas, em que o speckle está
totalmente desenvolvido, os retornos SAR são representados pela distribuição Gamma, e áreas heterogêneas requerem distribuições mais flexíveis para descrever
o espalhamento complexo, geralmente representado pela distribuição
$\mathcal{G}_I^0$.
Embora essa discriminação seja essencial para aplicações de sensoriamento remoto, 
testes paramétricos clássicos geralmente não são adequados para essa tarefa devido a 
limitações analíticas e numéricas.
Para superar esses desafios, propomos três estatísticas de teste para detectar heterogeneidade
em imagens SAR, com base nas entropias de Shannon, Rényi e Tsallis. Os testes associados utilizam
estimadores não paramétricos de entropia construídos a partir de espaçamentos amostrais,
evitando suposições explícitas sobre a distribuição subjacente. Para aumentar a precisão dos testes,
especialmente em amostras pequenas, incorporamos um procedimento de correção de viés via bootstrap,
que melhora a estabilidade dos estimadores, reduz o viés e o erro quadrático médio.
Os testes propostos são avaliados por meio de simulações de Monte Carlo,
tendo como critério de avaliação seu tamanho e poder sob diferentes condições de speckle e textura.
Os resultados mostram que os testes baseados nas entropias de Rényi e Tsallis superam a versão baseada
na entropia de Shannon, detectando variações de textura mais sutis e mantendo maior confiabilidade na
identificação de regiões verdadeiramente homogêneas.
Por fim, a metodologia é aplicada tanto a dados simulados como a dados SAR reais.
A análise é realizada com janelas deslizantes, gerando mapas de valores-$p$ que permitem a
avaliação visual e quantitativa da heterogeneidade espacial. O teste baseado na entropia de
Rényi mostra desempenho superior na identificação de padrões de rugosidade em pequena escala,
enquanto o teste baseado em Tsallis é melhor na detecção de regiões homogêneas. Em conjunto,
essas ferramentas baseadas em entropia oferecem uma estrutura robusta, interpretável e não
supervisionada para a detecção de heterogeneidade em dados SAR.

\vspace{1em}
\par
\noindent \textbf{Palavras-chave:} entropia;  heterogeneidade; distribuição gama; teste de hipótese; bootstrap.