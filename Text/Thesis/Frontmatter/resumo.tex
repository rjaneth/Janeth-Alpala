

Este trabalho apresenta uma abordagem estatística para detectar heterogeneidade em dados de 
intensidade de radar de abertura sintética (SAR) utilizando métodos baseados em entropia. 
Na análise de dados SAR, uma interpretação precisa do terreno depende fundamentalmente da 
distinção entre dois regimes principais: regiões homogêneas, caracterizadas por speckle 
totalmente desenvolvido e modeladas pela distribuição Gamma, e áreas heterogêneas, que 
exibem comportamentos de espalhamento complexos, geralmente representados pela distribuição 
$\mathcal{G}_I^0$.
Embora essa discriminação seja essencial para aplicações de sensoriamento remoto, 
testes paramétricos clássicos muitas vezes não a abordam de forma eficaz devido a 
limitações analíticas e instabilidades numéricas.
Para superar esses desafios, propomos três estatísticas de teste para detectar heterogeneidade 
em imagens SAR com base nas entropias de Shannon, Rényi e Tsallis. Esses testes utilizam 
estimadores não paramétricos de entropia construídos a partir de espaçamentos amostrais, 
evitando suposições explícitas sobre a distribuição subjacente. Para aumentar a precisão dos testes, 
especialmente em amostras pequenas, incorporamos um procedimento de correção de viés via bootstrap, 
que melhora a estabilidade dos estimadores, reduz o viés e o erro quadrático médio.
As estatísticas propostas são avaliadas por meio de simulações de Monte Carlo, 
onde analisamos seu tamanho e poder sob diferentes condições de speckle e textura. 
Os resultados mostram que os testes baseados em Rényi e Tsallis superam a versão baseada 
em Shannon, detectando variações de textura mais sutis e mantendo maior confiabilidade na 
identificação de regiões verdadeiramente homogêneas.
Por fim, a metodologia é aplicada a dados SAR simulados e reais. 
A análise é realizada com janelas deslizantes, gerando mapas de valores de $p$ que permitem a a
valiação visual e quantitativa da heterogeneidade espacial. O teste baseado na entropia de 
Rényi mostra desempenho superior na identificação de padrões de rugosidade em pequena escala, 
enquanto o teste baseado em Tsallis é eficaz na detecção de regiões homogêneas. Em conjunto, 
essas ferramentas baseadas em entropia oferecem uma estrutura robusta, interpretável e não 
supervisionada para a detecção de heterogeneidade em dados SAR.

\vspace{1em}
\par
\noindent \textbf{Palavras-chave:} Entropia; SAR; Heterogeneidade; Distribuição Gama; Teste de hipótese; Bootstrap