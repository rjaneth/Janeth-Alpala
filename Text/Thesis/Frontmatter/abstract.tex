%\begin{otherlanguage}{spanish}
%\begin{abstract}
%\phantomsection
%\addcontentsline{toc}{chapter}{Abstract}

This work presents a statistical approach to detect heterogeneity 
in synthetic aperture radar (SAR) intensity data using entropy-based methods.
In SAR data analysis, accurate terrain interpretation fundamentally depends 
on distinguishing two key regimes: homogeneous regions, characterized by fully 
developed speckle and modeled by the Gamma distribution, and heterogeneous areas, 
which exhibit complex scattering behaviors typically captured by the $\mathcal{G}_I^0$ distribution.
While this discrimination is essential for remote sensing applications, classical parametric tests often fail to address it 
effectively due to analytical limitations and numerical instabilities.
To overcome these challenges, we propose three test statistics for detecting heterogeneity 
in SAR imagery based on Shannon, Rényi, and Tsallis entropy measures. These tests rely on 
nonparametric entropy estimators constructed from sample spacings, avoiding explicit assumptions 
about the underlying distribution. To enhance their accuracy, particularly in small samples, 
we incorporate a bootstrap bias-correction procedure that improves the stability of the estimators, 
reduces bias, and lowers mean squared error.
The proposed tests are evaluated using Monte Carlo simulations, assessing size and power 
under varying sample sizes and numbers of looks.
Results demonstrate that the Rényi and Tsallis-based tests outperform the Shannon-based version by 
detecting more subtle texture variations while maintaining greater reliability in identifying 
truly homogeneous regions.
Finally, the methodology is applied to both simulated and real SAR data. We generate $p$-value maps using a 
sliding window analysis, allowing for visual and quantitative assessment of spatial heterogeneity. 
The Rényi-based test consistently identifies fine-scale roughness patterns, while the Tsallis-based 
test proves more effective in reliably detecting homogeneous areas. Together, these entropy-based 
tools provide a robust, interpretable, and fully unsupervised framework for heterogeneity detection 
in SAR data.

\vspace{1em}
\par
\noindent \textbf{Keywords:}  Entropy; SAR; Heterogeneity; Gamma distribution; Hypothesis testing; Bootstrap





%\end{otherlanguage}