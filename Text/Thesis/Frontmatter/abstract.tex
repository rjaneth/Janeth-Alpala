%\begin{otherlanguage}{spanish}
%\begin{abstract}
%\phantomsection
%\addcontentsline{toc}{chapter}{Abstract}

This work presents a statistical approach to detect heterogeneity 
in synthetic aperture radar (SAR) intensity data using entropy-based methods.
In SAR data analysis, an accurate interpretation of the terrain fundamentally 
depends on the distinction between two main regimes: homogeneous regions, in which 
speckle is fully developed and SAR returns are represented by the Gamma distribution, 
and heterogeneous areas, which require more flexible distributions to describe complex scattering, 
generally represented by the $\mathcal{G}_I^0$ distribution.
Although this discrimination is essential for remote sensing applications, 
classical parametric tests are generally not suitable for this task due to analytical and numerical limitations.
To overcome these challenges, we propose three test statistics to detect heterogeneity in SAR images based on 
Shannon, Rényi, and Tsallis entropies. 
The associated tests employ nonparametric entropy estimators constructed from sample spacings, avoiding explicit 
assumptions about the underlying distribution. To increase the accuracy of the tests, especially for small samples, 
we incorporate a bootstrap-based bias correction procedure that improves the stability of the estimators, reduces bias, 
and decreases the mean squared error.
The proposed tests are evaluated through Monte Carlo simulations, using test size and power under different speckle 
and texture conditions as performance criteria.
The results show that the tests based on Rényi and Tsallis entropies outperform the version based on Shannon entropy, 
detecting subtler texture variations and maintaining higher reliability in identifying truly homogeneous regions.
Finally, the methodology is applied to both simulated and real SAR data.
The analysis is performed using sliding windows, generating maps of $p$-values that allow for visual 
and quantitative assessment of spatial heterogeneity. 
The Rényi-based test consistently identifies fine-scale roughness patterns, while the Tsallis-based test performs better in detecting homogeneous regions. 
Together, these entropy-based tools provide a robust, interpretable, and unsupervised framework for detecting heterogeneity in SAR data.

\vspace{1em}
\par
\noindent \textbf{Keywords:}  entropy;  heterogeneity; gamma distribution; hypothesis testing; bootstrap.





%\end{otherlanguage}