%%%%%%%%%%%%%%%%%%%%%%%%%%%%%%%%%%%%%%%%%%%%%%%%%%%%%%%
%                   File: OSAmeetings.tex             %
%                  Date: 29 Novemver 2018              %
%                                                     %
%     For preparing LaTeX manuscripts for submission  %
%       submission to OSA meetings and conferences    %
%                                                     %
%       (c) 2018 Optical Society of America           %
%%%%%%%%%%%%%%%%%%%%%%%%%%%%%%%%%%%%%%%%%%%%%%%%%%%%%%%

\documentclass[letterpaper,10pt]{article} 
%% if A4 paper needed, change letterpaper to A4

\usepackage{./styles/osameet3} %% use version 3 for proper copyright statement

%% provide authormark
%\newcommand\authormark[1]{\textsuperscript{#1}}

%% standard packages and arguments should be modified as needed
\usepackage{amsmath,amssymb}
\usepackage[colorlinks=true,bookmarks=false,citecolor=blue,urlcolor=blue]{hyperref} %pdflatex
%\usepackage[breaklinks,colorlinks=true,bookmarks=false,citecolor=blue,urlcolor=blue]{hyperref} %latex w/dvipdf

\begin{document}

\title{Systematic Literature Reviews (SLR)}

% \author{Author name(s)}
% \address{Author affiliation and full address}
% \email{e-mail address}
%%Uncomment the following line to override copyright year from the default current year.
%\copyrightyear{2022}

\author{Janeth Alpala }

\address{  Universidade Federal de Pernambuco\\
}

%\email{\authormark{*}opex@optica.org} %% email address is required



%\begin{abstract}

%\end{abstract}

\section{Introduction}

Literature reviews have a fundamental role in academic research to assemble available knowledge and examine the state of a field.
Fink \cite{fink2019conducting} provides the following definition: ‘‘A literature review is a systematic, explicit, and reproducible design for identifying, evaluating, and interpreting the existing body of recorded documents’’.  Literature reviews generally serve two purposes: First, they summarize existing research by identifying patterns, themes, and issues. Second, it helps to identify the conceptual content of the field and can contribute to theory development.\\

Researchers typically collect available evidence on a topic or issue prior to conducting new research to assess the state of the
already available evidence. The idea behind a systematic review is to systematically collect available evidence and then provide an evaluation of the evidence based on predetermined criteria \cite{Tranfield2003}.

The different steps in conducting a systematic literature review include: identification of the literature to be included, data cleaning, analysis and synthesis, and presentation of the results \cite{Linnenluecke2019}.


\section{ Methodology: towards a systematic review}

The essential idea of a systematic literature review is that the review is replicable, which means that another researcher can replicate the review process and arrive at the same set of evidence and the same conclusion.

A systematic review includes an exhaustive search of the designated databases additional literature that may not be available through these databases and requires an exhaustive process to analyze and synthesize the relevant information.


\subsection{Identification of literature for inclusion}
An important step is to clarify the topic or issue under investigation. The researcher should investigate: Are there already existing reviews on the specific issue or research question under investigation? In this case, a replication might not be warranted, but an extension or update could prove useful.
\subsection{Data cleaning}
Once a range of suitable studies is identified, duplicates need to be removed from the analysis. In
addition, studies that are not relevant need to be removed. ‘False positives’ in the search process
can occur.


\subsection{Analysis and synthesis}

A vital step for any systematic review is the analysis and synthesis of the available evidence and depends on the number of studies that will be included in the review; the type of research method(s) used by individual studies and the quality of the evidence, and the chosen analytical or visualisation technique.


\subsection{Presentation of results}
There are numerous ways to present the results of a systematic literature review. If the studies
underlying the review use mainly qualitative data, the researcher can prepare a qualitative analysis,
but would not necessarily offer a statistical combination – aside from perhaps incorporating some
traditional descriptive statistics to summarise basic information, such as the
number of publications on a topic over time.\\

Alternative options for the presentation of results are offered by various software packages and
bibliographic mapping approaches that can be used for visualising research on a topic or theme.


\subsection{Material collection}

The first step consisting in the delimitation and compilation of the material. The literature search about edge detection was performed by considering scientific articles  in Web of Science (WoS) and Scopus, which are the two main bibliographic databases \cite{Pranckute2021}.

The employed keywords were the following focused on the fields Title-Abstract-Keywords:

\begin{itemize}
	\item (explainable edge detection)
\end{itemize}


\begin{longtable}{P{\dimexpr0.26\textwidth-2\tabcolsep-2\arrayrulewidth\relax}
                        P{\dimexpr0.39\textwidth-2\tabcolsep-\arrayrulewidth\relax}
                        P{\dimexpr0.30\textwidth-2\tabcolsep-\arrayrulewidth\relax}
                        %P{\dimexpr0.20\textwidth-2\tabcolsep-\arrayrulewidth\relax}
                        }
\toprule
Author & Title & Source \\\midrule
\rowcolor{gray!20}\multicolumn{3}{c}{Edge detection}\\
Nascimento, A. D., Horta, M. M., Frery, A. C., and Cintra, R. J. (2013)\cite{nascimento2013comparing}. & Comparing edge detection methods based on stochastic entropies and distances for PolSAR imagery & IEEE journal of selected topics in applied earth observations and remote sensing\\ \midrule
%\rowcolor{gray!20}\multicolumn{3}{c}{Music}\\
Xiang, Y., Wang, F., Wan, L., and You, H. (2017) \cite{xiang2017sar} & SAR-PC: Edge detection in SAR images via an advanced phase congruency model. & Remote Sensing\\ \midrule

Qin, X., Hu, T., Yu, W., Wang, P., Li, J.,  Zou, H. (2018) \cite{qin2018edge}& Edge detection of PolSAR images using statistical distance between automatically refined samples.& IEEE International Geoscience and Remote Sensing Symposium\\
\bottomrule
Quan, S., Xiang, D., Xiong, B.,  Kuang, G. (2020) \cite{quan2019edge}& Edge detection for PolSAR images integrating scattering characteristics and optimal contrast.& IEEE Geoscience and Remote Sensing Letters\\
\bottomrule
\caption{\label{Pesquisa de documentos.}}
\end{longtable}


%\begin{figure}[H]\vspace{-1em}
%\centering
% \includegraphics[width=1.0\textwidth]{ss.png}\vspace{-0.5em}
%\caption{\label{f1}}
%\vspace{-1em}
%    \end{figure}
%\begin{thebibliography}{99} %% use BibTeX or add references manually


\bibliographystyle{ieeetr}
\bibliography{../Common/references}  % e.g., \bibliography{cribari_01}
%\printbibliography




\end{document}
