%\begin{otherlanguage}{spanish}
\begin{abstract}
\phantomsection
\addcontentsline{toc}{chapter}{Abstract}
This work presents a statistical approach to identify the
underlying roughness characteristics in synthetic aperture radar (SAR)
intensity data. The physical modeling of this kind of data allows the
use of the Gamma distribution in the presence of fully-developed
speckle, i.e., when there are infinitely many independent backscatterers
per resolution cell, and none dominates the return. Such areas are often
called ``homogeneous'' or ``textureless'' regions. The
\(\mathcal{G}_I^0\) distribution is also a widely accepted law for
heterogeneous and extremely heterogeneous regions, i.e., areas where the
fully-developed speckle hypotheses do not hold.
One issue involving the parametric space of $\mathcal{G}_I^0$ is the analytical 
infeasibility of testing homogeneity against heterogeneity using classical tests. 
As solutions to this problem, we propose three test
statistics to distinguish between homogeneous and inhomogeneous regions,
i.e., between Gamma and \(\mathcal{G}_I^0\) distributed data, both with
a known number of looks. The first test statistic uses a bootstrapped
non-parametric estimator of Shannon entropy, providing an
assessment in uncertain distributional assumptions. The second test uses
the classical coefficient of variation (CV). The third test uses an
alternative form of estimating the CV based on the ratio of the mean
absolute deviation from the median to the median. We apply our test
statistic to create maps of \(p\)-values for the homogeneity hypothesis.
Finally, we show that our proposal, the entropy-based test, outperforms
existing methods, such as the classical CV and its alternative variant,
in identifying heterogeneity when applied to both simulated and actual
data.
\end{abstract}

\begin{keywords}
SAR; heterogeneity; entropy; coefficient of variation; hypothesis tests
\end{keywords}


%\end{otherlanguage}