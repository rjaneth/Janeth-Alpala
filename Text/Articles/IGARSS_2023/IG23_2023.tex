% Template for IGARSS-2020 paper; to be used with:
%          spconf.sty  - LaTeX style file, and
%          IEEEbib.bst - IEEE bibliography style file.
% --------------------------------------------------------------------------
\documentclass{article}
\usepackage{spconf,amsmath,epsfig}
\usepackage{bm,bbm}
\usepackage{float}
\usepackage{epstopdf} %para imagenes en eps y 
%%%
\usepackage{url}
\usepackage{booktabs,chemformula}
\addtolength{\tabcolsep}{-0.3em}
\usepackage[none]{hyphenat}
\usepackage{microtype}

\usepackage{subcaption}
\labelformat{subfigure}{(#1)}%() number
% --------------------

% --------------------
\def\x{{\mathbf x}}
\def\L{{\cal L}}

\DeclareMathOperator{\Tr}{Tr}

%


% Title.
% ------
\title{Quality assessment measures for explainable fusion of statistical evidences of edges in PolSAR images: A First Approach}
% Single address.
% ---------------

%\address{Author Affiliation(s)}
\name{Rosa Janeth Alpala$^a$, Anderson A.\ de Borba$^{b,c}$, and Alejandro C.\ Frery$^c$\thanks{e-mail:$^{a}$janeth.alpala@ufpe.br,$^{b}$anderson.borba@mackenzie.br, $^c$alejandro.frery@vuw.ac.nz}}
\address{$^a$ Universidade Federal de Pernambuco, 50740-540, Recife, PE, Brazil,           \\
$^b$Mackenzie Presbyterian University--UPM, FCI, BigMAAp, SP -- Brazil, \\
$^c$School of Mathematics and Statistics, Victoria University of Wellington, 6140, New Zealand.}

%
\begin{document}
%\ninept
%
\maketitle
%
\begin{abstract}

This work aims to develop techniques for the fusion of statistical evidences obtained from the application of Statistical Information Theory (SIT) and Statistical Information Geometry (SIG) in image processing and analysis, with a specific focus on Polarimetric Synthetic Aperture Radar (PolSAR) imagery. The goal is to generate a single solution superior to individual solutions. Furthermore, properties or measures that assess the quality of results will also be employed to evaluate the effectiveness of fusion methods in detecting edge evidence.

%% AAB: Olá Janeth, eu vou passar um verificador gramatical e ortográfico no texto.
% AAB: Está extranho a separação silábica nas palavras na quebra de linhas. Verifica como melhorar isso, é uma biblioteca não lembro qual. RSRSRSRSRS Talvez essa biblioteca \usepackage[english]{babel}. Mas, não tenho certeza. 
% AAB: A ideia acima não funcionou, pesquisar!!!!!
% RJA : corrigido melhorei com os pacotes  %\usepackage{microtype} e %\usepackage[none]{hyphenat}
\begin{keywords}
PolSAR, edge detection,  information fusion,  quality measures 
\end{keywords}
%
\end{abstract}

\vspace{-0.3cm}
\section{Introduction} \sloppy
\vspace{-0.2cm}
PolSAR imaging system captures valuable information in diverse conditions, including day and night, distinguishing it from optical sensors that depend on optimal weather conditions and daylight for effective operation~\cite{Hua2022}. It gains an advantage by utilizing four polarization combinations, specifically HH, HV, VH, and VV, based on horizontal and vertical polarization of transmitted and received signals. 
% AAB: Eu colocaria aqui informação das vantagens dos sensores PolSAR em realção a outros sensors, por exemplo,  sensores ópticos. 
% RJA : corrigido melhorei

Edge detection in PolSAR images is essential in many applications, such as speckle noise reduction, land-cover classification, and target recognition~\cite{Jin2016}. 

%AAB: Eu explicaria melhor como encontramos evidências de bordas. Eu pensaria melhor na palavra "deterministic".
% RJA : corrigido
The studies by de Borba et al.~\cite{DeBorba2020,FeatureSelectionforEdgeDetectioninPolSARImages} are the foundation for this research. The authors acquire numerous pieces of evidence regarding the location of edge points in PolSAR images. These estimations are then subjected to fusion processes to obtain an edge point that summarizes the evidence.
We extended this research by incorporating several quality metrics that not only assess the accuracy of the edge estimate, but also provide insight into the explainability of the fused edge evidence.
%%% ACF Não use números, use \ref
% RJA : corrigido

This paper is organized as follows:
Section \ref{sec_2} provides a brief introduction to the background knowledge, including PolSAR data and edge detection. Section \ref{sec_3} presents an overview of the quality measures employed. The evaluation of edge estimation quality using various metrics is discussed in Section \ref{sec_4}. Lastly, the conclusions are provided in Section \ref{sec_5}.


\section{Background}\label{sec_2}
\vspace{-0.2cm}
\subsection{PolSAR data representation}
Using PolSAR data, a wide range of biophysical and geophysical parameters related to the Earth's surface can be efficiently and reliably extracted. The polarimetric scattering matrix~\cite{Lee2017} serves as an instrumental tool to represent the polarimetric target information:
\begin{equation}
 \mathbf{S} = \begin{bmatrix}
S_{\text{HH}} & S_{\text{HV}} \\
S_{\text{VH}} & S_{\text{VV}}
\end{bmatrix},  
\label{E:a1}
\end{equation}
where the elements $S_{\text{HH}}$ and $S_{\text{VV}}$ are the returned power in the co-polarized channels, while the elements $S_{\text{HV}}$ and $S_{\text{VH}}$ relate to the cross-polarized channels.
If the PolSAR targets comply with the reciprocal condition $(S_{\text{HV}} = S_{\text{VH}})$, single-look PolSAR data can be represented using a scattering vector:
\begin{equation}
\mathbf{k}_s=\begin{bmatrix}
S_{\text{HH}} & \sqrt{2}S_{\text{HV}} & S_{\text{VV}}
\end{bmatrix}^t,
\label{E:21}
\end{equation}
% AAB: Cuidado!!! Você está usando s minusculo na eq acima e depois usa s maiusculo logo abaixo.
% RJA : corrigido
where  the superscript $t$ is the transpose operation. 
Multi-look PolSAR data can be expressed  by a covariance matrix  given as 
\(\mathbf{C}=\langle\mathbf{k}_S\mathbf{k}_S^\text{H} \rangle= {L}^{-1} \sum_{i=1}^{L} \mathbf{k}_S(i)\mathbf{k}_S(i)^\text{H} \), where $\langle \cdot \rangle$ is the ensemble average, $\text{H}$  the is complex conjugate transpose operator, and $L$ is the number of looks. 
The covariance matrix is hermitian, i.e., $\mathbf{C}= \mathbf{C}^\text{H}$, and positive definite;
see Ref.~\cite{Qin2022}.

\subsection{Gamma distribution}

We assume that the distribution of each intensity channel  follows a Gamma law, characterized by the probability density function
\begin{equation}
f_Z(z;\mu,L)=\frac{L^{L}z^{L-1}}{\mu^{L}\Gamma(L)} \exp\big\{-Lz/\mu\big\},\quad z>0,
\label{func_dens_uni_gamma}
\end{equation}
where $L>0$, and
$\mu>0$ is the mean.
The log-likelihood of the sample $\bm{z} = (z_1,\dots,z_n)$ under this model is
\begin{equation}
\mathcal{L}(\mu, L; \bm{z}) = 
n \big[L\ln (L / \mu) - \ln \Gamma(L)\big]
+L \sum_{k=1}^{n}\ln z_k -\frac{L}{\mu}\sum_{k=1}^{n} z_k.
\label{eq:LogLikelihoodGamma}
\end{equation}
Then, we find $\big(\widehat \mu, \widehat L\big)$, the maximum likelihood estimator of $(\mu, L)$ from $\bm{z}$, by maximizing~\eqref{eq:LogLikelihoodGamma}.


\subsection{Edge detection on a single data strip}
Finding the edges of an image is a crucial step in image analysis. These boundaries define distinct regions within the image, such as pastures, urban areas, or forested areas~\cite{monferran2020modelo}. Various techniques, including maximum likelihood, entropy, and geodesic distance, are commonly used in research studies to identify edge points~\cite{NaranjoTorres2017,Nascimento2019}. 
Suitable statistical models are needed for locating edge points in these types of images.
% AAB: Os dois paragrafos abaixo está redundante, tente reescrever fazendo uma fusãode informação entre eles. RSRSRS
% RJA : corrigido
The Gambini algorithm~\cite{Gambini2007} is a highly appealing method for edge detection. 
Data are collected within a narrow data strip to form the sample $\bm z = (z_1,z_2,\dots,z_n)$, which is partitioned at position $j$ into the interior $\bm z_\text{I}$ and exterior $\bm z_\text{E}$ samples:
$$
\bm z = (\underbrace{z_1,z_2,\dots,z_j}_{\bm z_\text{I}}, 
\underbrace{z_{j+1}, z_{j+2},\dots,z_n}_{\bm z_\text{E}}).
$$
Two possibly different models are assumed for each partition:
$\bm Z_\text{I} \sim \Gamma(\mu_\text{I},L_\text{I})$, and 
$\bm Z_\text{E} \sim \Gamma(\mu_\text{E},L_\text{E})$.
We then estimate $(\mu_\text{I},L_\text{I})$ and $(\mu_\text{E},L_\text{E})$ with $\bm z_\text{I}$ and $\bm z_\text{E}$, respectively, by maximizing~\eqref{eq:LogLikelihoodGamma}, and obtain $\big(\widehat{\mu}_\text{I}, \widehat{L}_\text{I}\big)$ and $\big(\widehat{\mu}_\text{E}, \widehat{L}_\text{E}\big)$.

Then, the total log-likelihood is
\begin{multline*}%\label{eq:TotalLogLikelihood}
\mathcal L\big(j;\widehat{\mu}_I, \widehat{L}_I,\widehat{\mu}_E, \widehat{L}_E\big)= -\Bigg(
	\frac{\widehat{L}_\text{I}}{\widehat{\mu}_\text{I}}\sum_{k=1}^{j} z_k +
	\frac{\widehat{L}_\text{E}}{\widehat{\mu}_\text{E}}\sum_{k=j+1}^{n} z_k  
	\Bigg)\mbox{}\\
+j \big[\widehat{L}_\text{I}\ln (\widehat{L}_\text{I} / \widehat{\mu}_\text{I}) - \ln \Gamma(\widehat{L}_\text{I})\big]
+\widehat{L}_\text{I} \sum_{k=1}^{j}\ln z_k  \mbox{}\\
+(n-j) \big[\widehat{L}_\text{E}\ln (\widehat{L}_\text{E} / \widehat{\mu}_\text{E}) - \ln \Gamma(\widehat{L}_\text{E})\big]
+\widehat{L}_\text{E} \sum_{k=j+1}^{n}\ln z_k .%-\\ 
\raisetag{2.2em}
\end{multline*}
The estimate of the edge position on the ray is the coordinate  $\widehat\jmath$ which maximizes $\mathcal L$.

\subsection{Fusion methods for edge evidence}
% AAB: Não estamos usando de fato o PCA, eu acho melhor escrever PCA based.
% AAB: Verifica no artigo (https://www.mdpi.com/2072-4292/15/9/2479) como nos referimos a fusão ROC, S-ROC. Talvez seja legal citar este artigo ele é mais atual. 
% RJA : corrigido
De Borba et al.~\cite{DeBorba2020,FeatureSelectionforEdgeDetectioninPolSARImages} fused the edges evidence to produce a unique and more accurate edge position estimator with six fusion methods: simple average, multiresolution discrete wavelet transform (MRDWT), principal component analysis (PCA) based method, receiver operating characteristic (S-ROC) statistics, multiresolution stationary wavelet transform (MR-SWT), and a multiresolution method based on singular value decomposition (MR-SVD).

Edge evidence fusion methods are essential to quantify and qualify the information obtained from each image channel. 
Such information enables the decision to utilize or discard data from a specific channel for improved edge detection.

\section{Quality assessment}\label{sec_3}

Commonly used quality measures for performance evaluation include Mean Absolute Error (MAE) and Root Mean Square Error (RMSE). MAE calculates the average absolute difference between estimated and reference values, being robust against outliers and independent of error signs. In contrast, RMSE emphasizes larger errors, making it more sensitive to outliers \cite{Ritter2013}. Additional precision evaluation measures include Standard Deviation (SD) and Correlation Coefficient (CC). SD measures the dispersion of values relative to their average, higher SD indicates lower precision. CC quantifies the linear relationship between the value sets~\cite{Vijayaraj}. The formulas are as follows:
\begin{align*}
	\text{RMSE} &=\sqrt{\frac{1}{n}\sum_{i=1}^{n}\left(y_i-\hat{y}_i\right)^2},\\
	\text{MAE} &=\frac{1}{n}\sum_{i=1}^{n}|y_i-\hat{y}_i|,\\
	\text{SD} &=\sqrt{\frac{1}{n-1}\sum_{i=1}^{n}\left(y_i-\bar{y}\right)^2},\\
\text{CC}&=  \frac{\sum_{i=1}^{n}(x_i-\Bar{x})(y_i-\Bar{y})}{\sqrt{\sum_{i=1}^{n}(x_i-\Bar{x})^2\sum_{i=1}^{n}(y_i-\Bar{y})^2}},
%\label{cc}
\end{align*}
where $n$ represents the number of observations, $y_i$ denotes the estimated values, $\hat{y}_i$ are the reference measured values, and  $\Bar{x}$ and $\Bar{y}$ are the mean values.
% AAB: Não está claro como você está usando a entropia
% Parece que você está usando de duas formas
% 1) Para calcular a entropia na figura 3. É uma ótima ideia e podendo ter uma boa medida % para definir se á borda ou não definindo um limiar
% 2) Você esta usando a entropia como um erro na tabela 1.
% Não esta bem explicado a diferênça entre as duas medidas.
% AAB: Outro ponto como você definiu a linha cor de laranja na figura 4. Descrever que é uma GR- Ground Reference
% RJA: ainda preciso melhorar essa ideia
Another metric that is used is the entropy, this quantifies the uncertainty of a probability distribution. In information theory, it reflects the level of uncertainty in a random experiment or signal. A peaked distribution has low entropy and low uncertainty, while a homogeneous distribution has high entropy and high uncertainty \cite{Mays2002}. For a discrete random variable $X$ with $\left\{x_1, x_2, \ldots, x_n\right\}$ the set of possible values it can take, Shannon entropy is defined as:
\begin{equation}
	H(X)=-\sum_{i=1}^{n}p(x_i)\log_2p(x_i),
	\label{eq:H}
\end{equation}
where $p(x_i)$ is the probability of the $i$th element.

%Entropy is commonly used in remote sensing to describe the information content of images. It employs expression \eqref{eq:H}  to assess the amount of information conveyed by each pixel within an image. 

\section{RESULTS} \label{sec_4}

Figs.~\ref{fig:all-ed}\ref{subfig:1ed}-\ref{subfig:3ed} show the edge evidences in the HH, HV, and VV channels obtained through maximum likelihood estimation.
De Borba et al.~\cite{DeBorba2020} applied various fusion methods to detect edge evidence. 
\begin{figure}[hbt]
    \begin{subfigure}{0.32\linewidth}
    \centering
    \includegraphics[width=\linewidth]{figures/hh_f.pdf}
    \caption{Channel HH.}
    \label{subfig:1ed}
  \end{subfigure}
  \begin{subfigure}{0.32\linewidth}
    \centering
    \includegraphics[width=\linewidth]{figures/hv.pdf}
    \caption{Channel HV.}
    \label{subfig:2ed}
  \end{subfigure}
  \begin{subfigure}{0.32\linewidth}
    \centering
    \includegraphics[width=\linewidth]{figures/vv_c.pdf}
    \caption{Channel VV.}
    \label{subfig:3ed}
  \end{subfigure}
  %\label{fig:1}
  \caption{Edges evidences from the three intensity channels.}
  \label{fig:all-ed}
\end{figure}
%\vspace{-0.4cm}
As an example, the results obtained by applying PCA, S-ROC, and MR-SWT methods to an AIRSAR L-band image of Flevoland are shown in Figs.~\ref{fig:all-fus}\ref{subfig:1fus}-\ref{subfig:3fus}.
%--------------------------------------------------
\begin{figure}[hbt]
    \begin{subfigure}{0.32\linewidth}
    \centering
    \includegraphics[width=\linewidth]{figures/roc.pdf}
    \caption{S-ROC fusion.}
    \label{subfig:1fus}
  \end{subfigure}
  \begin{subfigure}{0.33\linewidth}
    \centering
    \includegraphics[width=\linewidth]{figures/pca_1.pdf}
    \caption{PCA fusion.}
    \label{subfig:2fus}
  \end{subfigure}
  \begin{subfigure}{0.32\linewidth}
    \centering
    \includegraphics[width=\linewidth]{figures/swt.pdf}
    \caption{MR-SWT fusion.}
    \label{subfig:3fus}
  \end{subfigure}
  %\label{fig:1}
  \caption{Fusion methods.}
  \label{fig:all-fus}
\end{figure}
Figs.~\ref{fig:entropy1}\ref{subfig:1en}-\ref{subfig:3en}, illustrate the total log-likelihood $\mathcal L$ functions with respect to the $j$th position, where these functions exhibit peaks that potentially serve as indicators of edge position. Specifically, the concept of entropy is used as a means of evaluating the accuracy of edge position.

\begin{figure}[H]
\captionsetup[subfigure]{justification=centering}
   \centering
    \begin{subfigure}{0.49\linewidth}
    %\centering
    \includegraphics[width=4.2cm]{figures/likelihood_hh.pdf}
    \caption{\scriptsize{Channel HH (\textbf{Entropy = 0.0546}).}}
    \label{subfig:1en}
  \end{subfigure}
  \begin{subfigure}{0.49\linewidth}
   % \centering
    \includegraphics[width=4.2cm]{figures/likelihood_hv.pdf}
 \caption{\scriptsize{Channel HV (\textbf{Entropy =  0.6361}).}}
    \label{subfig:2en}
  \end{subfigure}\\
   \begin{subfigure}{0.6\linewidth}
    %\centering
    \includegraphics[width=4.5cm]{figures/likelihood_vv.pdf}
   % \centering
   \caption{\scriptsize{Channel VV (\textbf{Entropy = 0.8577}).}}
    \label{subfig:3en}
  \end{subfigure}
  \label{fig:1}
  \caption{Entropy as a quality measure for edge evidence.}
  \label{fig:entropy1}
\end{figure}
%--------------------------------------------------
To calculate entropy,  we employ the transformation of the total log-likelihood function into a discrete probability function. The sample is taken from a single data strip (ray) of length 120 pixels.  Subsequently, expression~\eqref{eq:H} is used to calculate the Shannon entropy, $H(X)$. The resulting value provides an objective metric for evaluating the accuracy of edge point detection. A low entropy value signifies precise estimation, characterized by a compact and peaked curve, see Fig.~\ref{fig:entropy1}\ref{subfig:1en}. Conversely, a high entropy value suggests less accurate estimation, featuring a more dispersed probability distribution and a graph with a flatter and elongated shape, see Fig.~\ref{fig:entropy1}\ref{subfig:2en}-\ref{subfig:3en}.

On the other hand, this study aims to determine whether evidence fusion in PolSAR images provides more information, leading to accurate edge detection, compared to edge detection of individual channels. To accomplish this, we use entropy as a measure of information content, in contrast to its usage as an estimation precision measure, as illustrated in Figs.~\ref{fig:entropy1}.

Table~\ref{tab_1} shows the evaluation metrics employed to assess the quality of the estimated edge points by comparing their similarity with a predefined set of reference points (ground reference). These metrics offer an objective evaluation of the accuracy of the obtained results. Entropy is calculated exclusively using the estimated edge point dataset ($\hat{x},\hat{y}$), while the remaining metrics (MAE, RMSE, SD, and CC) utilize both datasets. 
\begin{table}[hbt]
  \centering
  \begin{tabular}{@{}cccccc@{}}
    \toprule
Method  &Entropy    & MAE        & RMSE       &SD       & CC \\
    \midrule
   HH  & 6.6439      & 1.8400       & 3.7175   & 2.6291   & 0.9986 \\
   HV  & 6.6293      & 1.6700       & 2.9275   & 2.0640   & 0.9992 \\
   VV  & 6.6438      & 4.2400      & 7.9461    & 5.5649   & 0.9941 \\
   S-ROC & 6.0223      & 1.4154      & 2.3664    & 1.6311    & 0.9994 \\
   PCA & 7.7944      & 3.0180       & 6.0657   & 4.2788  & 0.9964 \\
   SWT & 12.0914     & 5.8006      & 9.2486   & 6.4161  & 0.9932 \\
    \bottomrule
  \end{tabular}\vspace{-0.1cm}
  \caption{Assessment metrics.}
  \label{tab_1}
  \end{table}

The importance of entropy  as a measure of information content is highlighted. For instance,  we observe low error and high correlation with the S-ROC method, indicating significant similarity between edge point positions. However, the low entropy value suggests a relatively small amount of information in the fused image. This implies precise edge detection with few outliers but exhibiting a sparse distribution of points, as shown in Fig.~\ref{F4}, where the orange rectangle indicates the ground reference.  \vspace{-0.45cm}
% AAB: Acho legal deixar claro que estes resultados são para a S-ROC
\begin{figure}[H] 
\centering
	\includegraphics[scale=0.5]{figures/roc_f.pdf}\vspace{-1.5mm} 
	\caption{Comparison of estimated points with ground reference using S-ROC method.}
	\label{F4}
\end{figure}
Meanwhile, the PCA based method demonstrates superior performance, as it exhibits a moderate entropy value along with low values of MAE, RMSE, and SD.

This approach enables a comprehensive evaluation of the characteristics and performance of fusion methods and individual channels. All metrics are depicted in Fig.~\ref{F5}. 

 \vspace{-0.45cm}
\begin{figure}[H] 
\centering
	\includegraphics[scale=0.5]{figures/metricas.pdf}
	\caption{Illustration of quality metrics.}\vspace{-0.3cm}
	\label{F5}
\end{figure}
The edge detection and fusion methods were implemented in Python. 
Code and data are available at \url{ https://github.com/rjaneth/code_igarss_23.git}

\section{CONCLUSION} \label{sec_5}

This paper presents some fusion methods for edge evidence and assesses their estimation precision using commonly employed metrics in the literature, with a specific emphasis on entropy as the primary measure.

By incorporating entropy as an evaluation metric, we enhance the explainability of our fusion methodology. Entropy provides a quantitative measure of the diversity and variability of the detected edges, allowing us to assess the amount of information captured by the fusion process. This transparency in evaluating the results enhances the interpretability and trustworthiness of our approach.

Overall, the results demonstrate the effectiveness of the proposed fusion methods for edge evidence and highlight the importance of incorporating explainability measures, such as entropy, in the evaluation process. These findings contribute to the field of image and data fusion, providing insights into the information content and accuracy of the detected edges.


% References should be produced using the bibtex program from suitable
% BiBTeX files (here: strings, refs, manuals). The IEEEbib.bst bibliography
% style file from IEEE produces unsorted bibliography list.
% ------------------------------------------------------------------------
%\bibliographystyle{IEEEtranS}
%\bibliographystyle{IEEEbib-abbrev}
\bibliographystyle{IEEEtran}
%\bibliography{strings,refs}

\bibliography{../../Common/references}
\end{document}
