% Template for IGARSS-2020 paper; to be used with:
%          spconf.sty  - LaTeX style file, and
%          IEEEbib.bst - IEEE bibliography style file.
% --------------------------------------------------------------------------
\documentclass{article}
\usepackage{spconf,amsmath,epsfig}
\usepackage{bm,bbm}
\usepackage{cite}
% \newcommand{\mcite}[1]{\mbox{\cite{#1}}}

% Example definitions.
% --------------------
\def\x{{\mathbf x}}
\def\L{{\cal L}}

\DeclareMathOperator{\Tr}{Tr}

% Title.
% ------
\title{Quality assessment measures for explainable fusion of statistical evidences of edges in PolSAR images: A First Approach}
% Single address.
% ---------------
%\name{Author(s) Name(s)\thanks{Thanks to XYZ agency for funding.}} 
%\address{Author Affiliation(s)}
\name{Rosa Janeth Alpala$^a$, Anderson A.\ de Borba$^{b,c}$, and Alejandro C.\ Frery$^c$.\thanks{e-mail:$^{a}$janeth.alpala@ufpe.br,$^{b}$anderson.borba@mackenzie.br, $^c$alejandro.freryorgambide@vuw.ac.nz }}
\address{$^a$ Universidade Federal de Pernambuco, 50740-540, Recife, PE, Brazil,           \\
$^b$Mackenzie Presbyterian University--UPM, FCI, BigMAAp, SP -- Brazil, \\
$^c$School of Mathematics and Statistics, Victoria University of Wellington, 6140, New Zealand.}

%
% For example:
% ------------
%\address{School\\
%	Department\\
%	Address}
%
% Two addresses (uncomment and modify for two-address case).
% ----------------------------------------------------------
%\twoauthors
%  {A. Author-one, B. Author-two\sthanks{Thanks to XYZ agency for funding.}}
%	{School A-B\\
%	Department A-B\\
%	Address A-B}
%  {C. Author-three, D. Author-four\sthanks{The fourth author performed the work
%	while at ...}}
%	{School C-D\\
%	Department C-D\\
%	Address C-D}
%
\begin{document}
%\ninept
%
\maketitle
%
\begin{abstract}
This work aims to develop techniques for the fusion of statistical evidences obtained from the application of Statistical Information Theory (SIT) and Statistical Information Geometry (SIG) in image processing and analysis, with a specific focus on Polarimetric Synthetic Aperture Radar (PolSAR) imagery. The goal is to generate a single solution superior to individual solutions. Furthermore, properties or measures that assess the quality of results will also be employed to evaluate the effectiveness of fusion methods in detecting edge evidence.

%
\begin{keywords}
PolSAR, edge detection,  information fusion,  quality
measures 
\end{keywords}
%
\end{abstract}
\section{Introduction}
PolSAR images offer a distinct advantage over other remote sensing images by utilizing four different polarization combinations, specifically HH, HV, VH, and VV~\cite{Hua2022}. These combinations are based on the horizontal (H), and vertical (V) polarizations of the received and transmitted signals. The polarization allows for the acquisition of detailed and comprehensive data, providing significantly more information than the single polarization mode~\cite{Zhai2015}.

The study on the applications of PolSAR data, including edge detection, has received increasing attention~\cite{Jin2016,Wang2018}. Edge detection in PolSAR images plays an essential role in many applications, such as speckle noise reduction, superpixel segmentation, land-cover classification, and target recognition~\cite{Xiang2016}. 
Several methods and algorithms have been proposed for edge detection in PolSAR imagery~\cite{Schou2003,Gambini2007,Nascimento2014}. 

Shi et al.~\cite{Shi2020} proposed a new method for extracting all edge-related features. 
This method is designed to reduce speckle noise in regions with both weak and strong edges, particularly in heterogeneous areas. 
Another approach~\cite{DeBorba2020} involves the fusion of evidence obtained from intensity channels. 
This can be done using a simple average, principal component analysis, ROC statistics, etc.

The study by de Borba et al.~\cite{DeBorba2020} is the foundation for this research. 
It finds evidence of the location of edge points by using a deterministic approach that does not take into account the variability of the estimator. 
We extended this investigation using new edge detectors~\cite{Xiang2016, Nascimento2014,Shi2020}. 
In addition, we employ various quality metrics to evaluate and compare the results of fusing edge evidence in PolSAR images.

\section{Background}
\subsection{PolSAR data representation}

Using PolSAR data, a wide range of biophysical and geophysical parameters related to the Earth's surface can be efficiently and reliably extracted. The polarimetric scattering matrix~\cite{Lee2017} serves as an instrumental tool to represent the polarimetric target information:
\begin{equation}
 \mathbf{S} = \begin{bmatrix}
S_{\text{HH}} & S_{\text{HV}} \\
S_{\text{VH}} & S_{\text{VV}}
\end{bmatrix},  
\label{E:a1}
\end{equation}
where the elements $S_{\text{HH}}$ and $S_{\text{VV}}$ are the returned power in the co-polarized channels, while the elements $S_{\text{HV}}$ and $S_{\text{VH}}$ relate to the cross-polarized channels.
If the PolSAR targets comply with the reciprocal condition $(S_{\text{HV}} = S_{\text{VH}})$, single-look PolSAR data can be represented using a scattering vector:
\begin{equation}
\mathbf{k}_s=[S_{\text{HH}}, \sqrt{2}S_{\text{HV}}, S_{\text{VV}} ]^t,
\label{E:21}
\end{equation}
where  the superscript $t$ stands for the transpose operation. 
Multi‑look PolSAR data can be expressed  by a covariance matrix  given as 
\(\mathbf{C}=\langle\mathbf{k}_S\mathbf{k}_S^\text{H} \rangle= {L}^{-1} \sum_{i=1}^{L} \mathbf{k}_s(i)\mathbf{k}_s(i)^\text{H} \), where $\langle \cdot \rangle$ is the ensemble average, $\text{H}$  the is complex conjugate transpose operator, and $L$ is the number of looks. 
The covariance matrix is hermitian, i.e., $\mathbf{C}= \mathbf{C}^\text{H}$, and positive definite;
see Ref.~\cite{Qin2022}.

\subsection{Gamma distribution}
%Over the past decades, several distributions have been created for modeling PolSAR image data. The most commonly used distribution for working with images generated from polarimetric multilook data is the complex Wishart distribution~\cite{Nielsen2007}. This distribution is characterized by two parameters, the covariance matrix $\mathbf{\Sigma}$ and the number of looks L of the image. Thus, the polarimetric matrix $\mathbf{Z} = \mathbf{C}=\left\langle\mathbf{k}_s\mathbf{k}_s^H \right\rangle$ can be derived following a complex Wishart distribution, represented as $\mathbf{Z}\sim \mathcal{W}_{\mathcal{C}}(\mathbf{\Sigma}, L),$  whose probability density function
%is~\cite{Frery2014, Qin2022}:
%\begin{align}
%f_{\mathbf{Z}}(\mathbf{Z})=\frac{L^{qL}|\mathbf{Z}|^{L-q}}{|\mathbf{\Sigma}|^L\Gamma_q(L)}\exp\left\{-L \Tr\left(\mathbf{\Sigma}^{-1}\mathbf{Z}\right)\right\},
%\label{E:22}
%\end{align}
%where $\Gamma_q(L)=\pi^{q(q-1)/2}\prod_{i=0}^{q-1}\Gamma(L-i)$,  $\Gamma(\cdot)$ denotes the gamma function and $\Tr(\cdot)$ is the trace of the matrix. In this case, three channels will be used, i.e., $q = 3$.

We assume that the distribution of each intensity channel  follows a Gamma law, characterized by the probability density function
\begin{equation}
f_Z(z;\mu,L)=\frac{L^{L}z^{L-1}}{\mu^{L}\Gamma(L)} \exp\big\{-Lz/\mu\big\},\quad z>0,
\label{func_dens_uni_gamma}
\end{equation}
where $L>0$, and
$\mu>0$ is the mean.
The log-likelihood of the sample $\bm{z} = (z_1,\dots,z_n)$ under this model is
\begin{equation}
\mathcal{L}(\mu, L; \bm{z}) = 
n \big[L\ln (L / \mu) - \ln \Gamma(L)\big]
+L \sum_{k=1}^{n}\ln z_k -\frac{L}{\mu}\sum_{k=1}^{n} z_k.
\label{eq:LogLikelihoodGamma}
\end{equation}
Then, we find $\big(\widehat \mu, \widehat L\big)$, the maximum likelihood estimator (MLE) of $(\mu, L)$ from $\bm{z}$, by maximizing~\eqref{eq:LogLikelihoodGamma}.

% A simplified derivation to obtain the complex Wishart distribution was presented in \cite{LEE1994}.
\subsection{Edge detection on a single data strip}
Finding the edges of an image is a crucial step in image analysis. These boundaries define distinct regions within the image, such as pastures, urban areas, or forested areas~\cite{monferran2020modelo}. Various techniques, including maximum likelihood, entropy, and geodesic distance, are commonly used in research studies to identify edge points~\cite{NaranjoTorres2017,Nascimento2019}. 
Suitable statistical models are needed for locating edge points in these types of images.

The Gambini algorithm~\cite{Gambini2007} is a highly appealing method for edge detection. 
It searches for evidence of edges within a narrow data strip. 
The algorithm involves projecting rays and identifying edge evidence by maximizing a value function.

The algorithm starts by casting rays from a point inside the candidate region, e.g., the centroid.
Data are collected around each ray to form the sample $\bm z = (z_1,z_2,\dots,z_n)$, which is partitioned at position $j$ into the interior $\bm z_\text{I}$ and exterior $\bm z_\text{E}$ samples:
$$
\bm z = (\underbrace{z_1,z_2,\dots,z_j}_{\bm z_\text{I}}, 
\underbrace{z_{j+1}, z_{j+2},\dots,z_n}_{\bm z_\text{E}}).
$$
Two possibly different models are assumed for each partition:
$\bm Z_\text{I} \sim \Gamma(\mu_\text{I},L_\text{I})$, and 
$\bm Z_\text{E} \sim \Gamma(\mu_\text{E},L_\text{E})$.
We then estimate $(\mu_\text{I},L_\text{I})$ and $(\mu_\text{E},L_\text{E})$ with $\bm z_\text{I}$ and $\bm z_\text{E}$, respectively, by maximizing~\eqref{eq:LogLikelihoodGamma}, and obtain $\big(\widehat{\mu}_\text{I}, \widehat{L}_\text{I}\big)$ and $\big(\widehat{\mu}_\text{E}, \widehat{L}_\text{E}\big)$.

Then, the total log-likelihood is
\begin{equation}\label{eq:TotalLogLikelihood}
\begin{aligned}
\mathcal L\big(j&;\widehat{\mu}_I, \widehat{L}_I,\widehat{\mu}_E, \widehat{L}_E\big)= -\Bigg(
	\frac{\widehat{L}_\text{I}}{\widehat{\mu}_\text{I}}\sum_{k=1}^{j} z_k +
	\frac{\widehat{L}_\text{E}}{\widehat{\mu}_\text{E}}\sum_{k=j+1}^{n} z_k  
	\Bigg)\mbox{}\\
&+j \big[\widehat{L}_\text{I}\ln (\widehat{L}_\text{I} / \widehat{\mu}_\text{I}) - \ln \Gamma(\widehat{L}_\text{I})\big]
+\widehat{L}_\text{I} \sum_{k=1}^{j}\ln z_k  \mbox{}\\
&+(n-j) \big[\widehat{L}_\text{E}\ln (\widehat{L}_\text{E} / \widehat{\mu}_\text{E}) - \ln \Gamma(\widehat{L}_\text{E})\big]
+\widehat{L}_\text{E} \sum_{k=j+1}^{n}\ln z_k .%-\\ 
\raisetag{2.2em}
\end{aligned}
\end{equation}
The estimate of the edge position on the ray is the coordinate  $\widehat\jmath$ which maximizes $\mathcal L$.

\subsection{Fusion methods for edge evidence}

De Borba et al.~\cite{DeBorba2020} fused the edges evidence obtained in intensities channels HH, HV, and VV to produce a unique and more accurate edge position estimator. 
The authors used six fusion methods: simple average, multiresolution discrete wavelet transform (MRDWT), principal component analysis (PCA), receiver operating characteristic (ROC) statistics, multiresolution stationary wavelet transform (MR-SWT), and a multiresolution method based on singular value decomposition (MR-SVD).

Edge evidence fusion methods are essential to quantify and qualify the information obtained from each image channel. 
Such information enables the decision to utilize or discard data from a specific channel to enhance edge detection accuracy.

\section{Quality assessment}

Some of the most frequently used indicators of quality are the mean square error (MSE)~\cite{beaulieu2003multi}, 
the correlation coefficient (CC)~\cite{aiazzi2004spectral}, and the entropy (H)~\cite{han2008study}:
\begin{itemize}
	\item Mean square error measures the spectral distortion introduced by the fusion process.
	$$MSE= \displaystyle \frac{\sum_{i=1}^{M}\sum_{i=1}^{N}(f(i,j)-r(i,j))^2}{M\times N}.$$
	\item The correlation coefficient measures the degree of correlation between the enhanced and the original images.
	$$CC=  \frac{\sum_{i=1}^{n}(x_i-\Bar{x})(y_i-\Bar{y})}{\sqrt{\sum_{i=1}^{n}(x_i-\Bar{x})^2(y_i-\Bar{y})^2}}.$$
	\item The entropy is used for measuring the richness of data in enhanced images.
	$$H=-\sum_{i=1}^{n}p(i)\log_2p(i).$$
	%%% ACF O que é d(i)?
\end{itemize}

We follow the image fusion quality assessment strategy proposed in~\cite{Li2010, Somvanshi2017,Nikola2015} for the problem at hand, using statistical analyses.

%We use the criteria described in [2]  to evaluate the quality of each evidence for each fusion method.

%We expanded our research to a statistical analysis, using quality metrics to assess the efficiency in the fusion of evidence, using different methods. 

\subsection{Exploring the concept of explainability}

Explainability is crucial when it comes to artificial intelligence (AI) and related fields~\cite{Vilone2021}. 
It refers to the ability to understand and interpret the decisions and actions made by an automatic system. 

The proposed methodology in this work focuses on identifying interpretable measures that can be used to weigh the influence of individual estimates during an evidence fusion process.

\section{RESULTS}

Figs.~\ref{fig:res}(a)-(c) show the edge evidences in the HH, HV and VV channels obtained through maximum likelihood estimation.
De Borba et al.~\cite{DeBorba2020} applied various fusion methods to detect edge evidence. 
As an example, the results obtained by applying PCA and ROC methods to an AIRSAR L-band image of Flevoland are shown in Fig.~\ref{fig:2}.%(a)-(b).

%%% The source is in ../../../Figures/PDF/hh.pdf
 \begin{figure}[htb]
\begin{minipage}[b]{1.0\linewidth}
  \centering
 \centerline{\epsfig{figure=../../../Figures/PDF/hh.pdf,width=4cm}}
  \vspace{-0.1cm}
  \centerline{(a) Channel HH.}\medskip
\end{minipage}
%
\begin{minipage}[b]{.48\linewidth}
  \centering
\centerline{\epsfig{figure=../../../Figures/PDF/hv.pdf,width=4.0cm}}
  \vspace{-0.1cm}
  \centerline{(b) Channel HV.}\medskip
\end{minipage}
\hfill
\begin{minipage}[b]{0.48\linewidth}
  \centering
 \centerline{\epsfig{figure=../../../Figures/PDF/vv.pdf,width=4.0cm}}
  \vspace{-0.1cm}
  \centerline{(c) Channel VV.}\medskip
\end{minipage}
%
\caption{Edges evidences from the three intensity channels.}
\label{fig:res}
%
\end{figure}
 \begin{figure}[htb]
\begin{minipage}[b]{.48\linewidth}
  \centering
\centerline{\epsfig{figure=../../../Figures/PDF/pca.pdf,width=4.0cm}}
  \vspace{-0.1cm}
  \centerline{(a) PCA fusion.}\medskip
\end{minipage}
\hfill
\begin{minipage}[b]{0.48\linewidth}
  \centering
 \centerline{\epsfig{figure=../../../Figures/PDF/roc.pdf,width=4.0cm}}
  \vspace{-0.1cm}
  \centerline{(b) ROC fusion.}\medskip
\end{minipage}
%
\caption{Results of applying the two fusion methods.}
\label{fig:2}
%
\end{figure}

\section{CONCLUSION}
The project outlined here will develop fusion techniques that factor in the quality of each evidence. In this way, evidence subjected to great variability or that may be inconsistent with other pieces of evidence, will have less influence.

An explainable fusion of statistical evidence can provide semantically rich information on the contribution of each component to the composition of the final result. 
This enables the extraction of relevant and dependable information about the content of the evidence from each source. 

Finally, we will use several evaluation indicators presented in the literature to evaluate the quality of edge evidence fusion in PolSAR images.

%(bias, variance, etc.)
%Lastly, we expanded our research to include statistical analysis by utilizing quality metrics() to evaluate the effectiveness of the fusion of evidence.

%\newpage



% % -------------------------------------------------------------------------
% \vfill

%  \textbf{Instruction for References:} List and number all bibliographical references at the end of the paper.  The references can be numbered in alphabetic order or in order of appearance in the document.  When referring to them in the text, type the corresponding reference number in square brackets as shown at the end of this sentence \cite{C2}.
% % -------------------------------------------------------------------------



%\pagebreak



% References should be produced using the bibtex program from suitable
% BiBTeX files (here: strings, refs, manuals). The IEEEbib.bst bibliography
% style file from IEEE produces unsorted bibliography list.
% ------------------------------------------------------------------------
%%% The source is in ../../Common/references
\bibliographystyle{IEEEtranS}
\bibliography{../../Common/references}
\end{document}
